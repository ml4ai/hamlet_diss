\documentclass[11pt, serif, mathserif, table,trans]{beamer}

% ================== PACKAGES ==================
\usepackage{amsmath}
\usepackage{amssymb}
\usepackage{amsthm}
\usepackage{array}
\usepackage{bbm}
\usepackage{caption}
\usepackage{color}
\usepackage{fancyvrb}
\usepackage[T1]{fontenc}
% \usepackage[pdftex]{graphicx}
% \usepackage{handoutWithNotes}
\usepackage{hyperref}
\usepackage{multirow}
\usepackage{nicefrac}
\usepackage{subcaption}
\usepackage{tikz}
\usepackage{qtree}
\usepackage{wrapfig}
\usepackage{xcolor}
\usetikzlibrary{arrows,shapes}

%%%%%%%%%%%%%%%%%%%%%%%%%%%%% Useful math macros %%%%%%%%%%%%%%%%%%%%%%%%%%%%%%%
% argmax and argmin
\DeclareMathOperator*{\argmax}{arg\,max}
\DeclareMathOperator*{\argmin}{arg\,min}

%% Distributions
\newcommand{\Norm}[2]{\mathcal{N}(#1,#2)}
\newcommand{\Unif}{\mathsf{U}}
\newcommand{\Pois}[1]{\mathsf{Poisson}(#1)}
\newcommand{\Exp}[1]{\mathsf{Exponential}(#1)}
\newcommand{\Gamm}[2]{\mathsf{Gamma}(#1,#2)}
\newcommand{\Bern}[1]{\mathsf{Bernoulli}\left(#1\right)}
\newcommand{\Binom}[2]{\mathsf{Binom}(#1,#2)}
\newcommand{\Geom}[1]{\mathsf{Geometric}(#1)}
\newcommand{\Cat}{\mathsf{Categorical}}
\newcommand{\Beta}[2]{\mathsf{Beta}(#1,#2)}
\newcommand{\Lapl}{\mathsf{Laplace}}
\newcommand{\GP}{\mathcal{GP}}
\newcommand{\DP}[1]{\mathsf{DP}(#1)}
\newcommand{\CRP}[1]{\mathsf{CRP}(#1)}
\newcommand{\GEM}[1]{\mathsf{GEM}(#1)}
\newcommand{\Dir}[1]{\mathsf{Dirichlet}(#1)}
\newcommand{\Discrete}[1]{\mathsf{Discrete}(#1)}

%% Probability
\newcommand{\E}[1]{\mathbb{E}[#1]}
\newcommand{\V}[1]{\mathbb{V}[#1]}
\newcommand{\Cov}[2]{\mathbb{C}\mathrm{ov}(#1,#2)}
\newcommand{\given}{\, \vert \,}

%% General
\newcommand{\R}{\mathbb{R}}
\newcommand{\Z}{\mathbb{Z}}
\newcommand{\N}{\mathbb{N}}
\newcommand{\abs}[1]{\left\vert #1 \right\vert}
\newcommand{\norm}[1]{\left\vert \left \vert #1 \right\vert \right\vert}

%% Vectors and Matrices
\newcommand{\bx}{\mathbf{x}}
\newcommand{\bX}{\mathbf{X}}
\newcommand{\by}{\mathbf{y}}
\newcommand{\bY}{\mathbf{Y}}
\newcommand{\ba}{\mathbf{a}}
\newcommand{\bA}{\mathbf{A}}
\newcommand{\bb}{\mathbf{b}}
\newcommand{\bB}{\mathbf{B}}
\newcommand{\bc}{\mathbf{c}}
\newcommand{\bM}{\mathbf{M}}
\newcommand{\bC}{\mathbf{C}}
\newcommand{\bQ}{\mathbf{Q}}
\newcommand{\bl}{\mathbf{l}}
\newcommand{\bem}{\mathbf{m}}
\newcommand{\bq}{\mathbf{q}}
\newcommand{\bz}{\mathbf{z}}
\newcommand{\bZ}{\mathbf{Z}}
\newcommand{\bu}{\mathbf{u}}
\newcommand{\br}{\mathbf{r}}
\newcommand{\bs}{\mathbf{s}}
\newcommand{\bw}{\mathbf{w}}
\newcommand{\bW}{\mathbf{W}}
\newcommand{\bT}{\mathbf{T}}
\newcommand{\bth}{\bm\uptheta}
\newcommand{\bTh}{\bm\Theta}
\newcommand{\bbeta}{\boldsymbol{\beta}}
\newcommand{\btheta}{\boldsymbol{\theta}}
\newcommand{\bpi}{\boldsymbol{\pi}}
\newcommand{\bphi}{\boldsymbol{\phi}}
\newcommand{\btau}{\boldsymbol{\tau}}
\newcommand{\bmu}{\boldsymbol{\mu}}
\newcommand{\blambda}{\boldsymbol{\lambda}}
\newcommand{\bSigma}{\boldsymbol{\Sigma}}

%% Sets
\newcommand{\cA}{\mathcal{A}}
\newcommand{\cB}{\mathcal{B}}
\newcommand{\cZ}{\mathcal{Z}}
\newcommand{\cM}{\mathcal{M}}
% \usepackage{mathpazo}

% \mode<handout>{
% \pgfpagesuselayout{3 on 1 with notes}[letterpaper, border, shrink=0mm]
% }

\definecolor{paperbg}{HTML}{FDF6E3}
\definecolor{paperbgdark}{HTML}{EDE6CA}
\definecolor{solarred}{HTML}{DC322F}
\definecolor{solarlightorange}{HTML}{EBD6C3}
\definecolor{solarlightgreen}{HTML}{F3FFC8}
\definecolor{solarorange}{HTML}{C45B2A}
\definecolor{solaryellow}{HTML}{A57900}
\definecolor{solargreen}{HTML}{859900}
\definecolor{solarcyan}{HTML}{2AA198}
\definecolor{solarblue}{HTML}{268BD2}
\definecolor{solardarkblue}{HTML}{005BA2}
\definecolor{solarviolet}{HTML}{6C71C4}
\definecolor{solarmagenta}{HTML}{C05068}
\definecolor{solarbase2}{HTML}{EEE8D5}
\definecolor{solarbase1}{HTML}{93A1A1}
\definecolor{solarbase00}{HTML}{657B83}
\definecolor{solarbase01}{HTML}{586E75}
\definecolor{solarbase015}{HTML}{174652}
\definecolor{solarbase02}{HTML}{073642}
\definecolor{solarbase03}{HTML}{002B36}
\definecolor{functionblue}{HTML}{40A0B8}
\definecolor{solarlightblue}{HTML}{D9EEEF}

% \usetheme{Dresden}

\setbeamercolor{background canvas}{bg=paperbg}
% \mode<handout>{
% \setbeamercolor{background canvas}{bg=white}
% }
\setbeamercolor{normal text}{fg=solarbase01}
\usecolortheme[named=functionblue]{structure}
\setbeamercolor{title in sidebar}{fg=solaryellow}
\setbeamercolor{author in sidebar}{fg=solaryellow}
\setbeamercolor{section in head/foot}{fg=paperbg}
\setbeamercolor{section in sidebar}{fg=functionblue}
\setbeamercolor{section in sidebar shaded}{fg=solarbase1}

\usecolortheme{orchid}
\setbeamercolor{block body}{fg=solarbase01, bg=solarlightgreen}
\setbeamercolor{block title}{fg=paperbg, bg=solargreen}
\setbeamercolor{alertblock body}{fg=solarbase01, bg=solarlightorange}
\setbeamercolor{alertblock title}{fg=paperbg, bg=solarorange}
\setbeamertemplate{navigation symbols}{}

\newenvironment<>{attnblock}[1]{%
  \begin{actionenv}#2%  
      \def\insertblocktitle{#1}%
      \par%
      \mode<presentation>{%
        \setbeamercolor{block title}{fg=paperbg, bg=solarorange}
        \setbeamercolor{block body}{fg=solarbase01, bg=solarlightorange}
        \setbeamercolor{itemize item}{fg=solarorange}
        \setbeamercolor{itemize subitem}{fg=solarorange}
        }%
      \usebeamertemplate{block begin}}
{\par\usebeamertemplate{block end}\end{actionenv}}

\newenvironment<>{infoblock}[1]{%
  \begin{actionenv}#2%  
      \def\insertblocktitle{#1}%
      \par%
      \mode<presentation>{%
        \setbeamercolor{block title}{fg=paperbg, bg=solargreen}
        \setbeamercolor{block body}{fg=solarbase01, bg=solarlightgreen}
        \setbeamercolor{itemize item}{fg=solargreen}
        \setbeamercolor{itemize subitem}{fg=solargreen}
        \setbeamercolor{enumerate item}{fg=solargreen}
        }%
      \usebeamertemplate{block begin}}
{\par\usebeamertemplate{block end}\end{actionenv}}

\newenvironment<>{exblock}[1]{%
  \begin{actionenv}#2%  
      \def\insertblocktitle{#1}%
      \par%
      \mode<presentation>{%
        \setbeamercolor{block title}{fg=paperbg, bg=functionblue}
        \setbeamercolor{block body}{fg=solarbase01, bg=solarlightblue}
        \setbeamercolor{itemize item}{fg=functionblue}
        \setbeamercolor{itemize subitem}{fg=functionblue}
        }%
      \usebeamertemplate{block begin}}
{\par\usebeamertemplate{block end}\end{actionenv}}

% Double-line for start and end of epigraph.
\newcommand{\epiline}{\hrule \vskip -.2em \hrule}
% Massively humongous opening quotation mark.
\newcommand{\hugequote}{%
  \fontsize{42}{48}\selectfont \color{solarcyan} {\rm \textbf{``}}
  \vskip -.5em
}

% Beautify quotations.
\newcommand{\epigraph}[3]{%
  \bigskip
  \colorbox{solarbase2}{%
    \parbox{#1\textwidth}{%
    \epiline \vskip 1em {\hugequote} % \vskip -.5em
    % \parindent 2.2em
    {\rm #2}\begin{flushright}{\rm  #3}\end{flushright}
    \epiline
    }
  }
  \bigskip
}

\title{Modeling and Unsupervised Learning of Structured Similarity Among
  Source Contexts in Bayesian Hierarchical Infinite Mixture Models}
\subtitle{With Two Applications to Modeling Natural Language Semantics}
\author{Colin Reimer Dawson}
\date{\today}

\begin{document}

\frame{\titlepage}

\bibliographystyle{apalike}

\begin{frame}
  \frametitle{Outline}
  \tableofcontents
\end{frame}

%====================================================
\section{Background and Problem Formulation}
\subsection{Mixture Models}
\label{sec:mixture-models}

\begin{frame}
  \frametitle{Mixture Models}  
  \begin{itemize}[<+->]
  \item Goal: Estimate unknown density, $f$, of the form
    \begin{equation}
      \label{eq:3}
      f(\bx) = \sum_{k} \pi_k f_k(\bx)
    \end{equation}
  \item Traditionally, number of components is fixed and $f_k$ have parametric form:
    \begin{equation}
      \label{eq:4}
      f(\bx; \bpi, \btheta) = \sum_{k=1}^K \pi_k f(\bx; \theta_k)
    \end{equation}
  \item Estimate $\bpi$ and $\btheta$.
    \begin{itemize}
    \item MLE: can find local optimum using Expectation-Maximization
      (EM) algorithm.
    \end{itemize}
  \end{itemize}
\end{frame}
%---------------------------------------------------

\begin{frame}
  \frametitle{A Bayesian Approach}
  \begin{itemize}
  \item Standard Bayesian version:
    \begin{align}
      \label{eq:5}
      \pi &\sim \mathrm{Dirichlet}(\alpha\mathbf{1}_K) \\
      \theta_k &\stackrel{i.i.d.}{\sim} f\text{-}\mathrm{Conjugate}(\xi)
    \end{align}
    \item Straightforward to do Gibbs Sampling
  \end{itemize}
\end{frame}

%---------------------------------------------------
\subsection{Infinite Mixture Models}
\label{sec:dirichl-proc-mixt}
\begin{frame}
  \frametitle{Unbounded number of components}
  \begin{itemize}
  \item Having to specify $K$ in advance is limiting.  Too high $\to$
    overfitting.  Too low $\to$ underfitting.
  \item We can instead use an {\it infinite mixture model}, with a
    prior to guard against overfitting.
  \end{itemize}
\end{frame}

\begin{frame}
  \frametitle{Dirichlet Processes}
  \begin{infoblock}{Definition: Dirichlet Process}
    A {\bf Dirichlet Process} \cite{ferguson1973bayesian} with {\bf
      base probability measure} $G_0$ and {\bf concentration
      parameter} $\alpha > 0$ is a random measure, $\mu$ on a measure
    space $(\cX, \Sigma)$ with the property that, for any finite
    partition, $\{A_1, \dots, A_n\}$ of $\cX$, the induced random
    vector
    \begin{equation}
      \label{eq:6}
      (\mu(A_1), \dots, \mu(A_n)) \sim
      \mathrm{Dirichlet}(\alpha G_0(A_1), \dots, \alpha G_0(A_n))
    \end{equation}
  \end{infoblock}
  \begin{itemize}
  \item Note: $\mu$ is atomic $a.s.$.
  \item Note: If $G_0$ is atomic with finite support, $\mu$ reduces to a
    Dirichlet distribution over those atoms.
  \end{itemize}
\end{frame}

\begin{frame}
  \frametitle{Normalized Gamma Process Representation}
  The DP is obtained by normalizing a {\it Gamma Process}:

  A {\bf Gamma Process} is a Poisson Process on $\mathbb{R}^{+} \times
  \Theta$ with L\'evy intensity measure
  \begin{equation}
    \label{eq:8}
    \nu(d\pi, d\theta) = \alpha \pi^{-1} e^{-\pi} d\pi G_0(d\theta)
  \end{equation}

  That is, consider a random collection of point masses $\{\theta_k\}$
  on $\Theta$, with respective random masses $\{\pi_k\}$ as points
  $(\pi_k,\theta_k) \in \mathbb{R}^+ \times \Theta$.  The number $n(A)$
  of such points in a region $A \subset \mathbb{R}^+ \times \Theta$ is
  distributed
  \begin{equation}
    \label{eq:9}
    n(A) \sim \Pois{\int_A \nu(d\pi, d\theta)}
  \end{equation}
  
\end{frame}

\begin{frame}
  \frametitle{Normalized Gamma Process Representation}
  
  The sum $T = \sum_{k} \pi_k$ is finite almost surely (see, e.g., \cite{ferguson1973bayesian}), so we can
  normalize the set of atoms to form a probability measure on $\Theta$.

  This normalized measure is distributed $\mathrm{DP}(\alpha G_0)$.
\end{frame}

\begin{frame}
  \frametitle{A Constructive Definition of the DP}
  \begin{infoblock}{The Stick-Breaking Construction \cite{sethuraman1991constructive}}
    Define 
    \begin{align}
      \{\pi'_k\}_{k=1}^{\infty} &\stackrel{i.i.d}{\sim} \Beta{1}{\alpha}
      \\
      \pi_k &= \pi'_k \prod_{k=1}^{k-1} (1 - \pi'_{k-1}) \\
      \{\theta_k\}_{k=1}^{\infty} &\stackrel{i.i.d.}{\sim} G_0
    \end{align}
    Then
    \begin{equation}
      \label{eq:7}
      \mu \stackrel{def}{=} \sum_{k=1}^{\infty} \pi_k \delta_{\theta_k}
    \end{equation}
    is distributed as a Dirichlet Process with base measure $G_0$ and
    concentration parameter $\alpha$.
\end{infoblock}
\end{frame}

%---------------------------------------------------
\begin{frame}
  \frametitle{An Infinite Mixture Model}
  \begin{infoblock}{DP Mixture Model}
  If we let
  \begin{align}
    \label{eq:10}
    \bpi &\sim \mathrm{Stick}(\alpha) \\
    \{\theta_k\} &\stackrel{i.i.d}{\sim} G_0 \\
    f(\bx \given \bpi, \btheta) &= \sum_{k=1}^\infty \pi_k f(\bx \given
    \theta_k)
  \end{align}
  then $\bx$ are distributed according to a {\bf Dirichlet Process
    Mixture Model}
\end{infoblock}
\end{frame}

\begin{frame}
  \frametitle{Infinite Gaussian Mixture Model}
  For example, if $f(\bx \given \theta_k)$ is a Normal density, we
  have the {\bf Infinite Gaussian Mixture model}
  \cite{rasmussen2000infinite}.

  \vspace{0.2in}

   We could then let $G_0$ be a Normal Inverse Gamma model for $\mu$
   and $\sigma^2$.
\end{frame}

%---------------------------------------------------
\subsection{Hierarchical Dirichlet Process Mixtures}
\label{sec:dirichl-proc-mixt}
\begin{frame}
  \frametitle{Hierarchical Dirichlet Processes}
  \begin{itemize}[<+->]
  \item If we have data from multiple sources, $j = 1, \dots, J$,
    whose generating distributions, $\{G_j\}$ are distinct but related, we can
    use a hierarchical prior to couple them, e.g.,
    \begin{equation}
      \label{eq:11}
      G_j \stackrel{i.i.d}{\sim} \mathrm{DP}(\alpha G_0),
    \end{equation}
    where dependence is introduced by putting a hyperprior on $G_0$
    and integrating it out.
  \item Problem: If $G_0$ is absolutely continuous, the atoms in the
    $G_j$ will be at disjoint locations.
  \item Solution: Let $G_0$ itself have a DP prior.
  \end{itemize}
\end{frame}

%---------------------------------------------------
\begin{frame}
  \frametitle{Hierarchical Dirichlet Processes}
    \begin{infoblock}{The Hierarchical Dirichlet Process Mixture Model
        \cite{teh2006hierarchical}}
      Define
      \begin{align}
        G_0 &\sim \mathrm{DP}(\gamma H) \\
        G_j \given G_0 &\stackrel{i.i.d}{\sim} \mathrm{DP}(\alpha G_0) \qquad \qquad \quad j = 1, \dots, J\\
        \{(\pi_{jk}, \theta_{jk})\}_{k=1}^{\infty} &\leftarrow G_j \\
        \{\bx_{jn}\}_{n=1}^N \given \{\pi_{jk} \theta_{jk}\}
        &\stackrel{i.i.d}{\sim} \sum_{k=1}^{\infty} \pi_{jk}
        f(\bx \given \theta_{jk})
      \end{align}
      This defines a Hierarchical Dirichlet Process (HDP) Mixture.
      Atoms are shared among contexts by virtue of the discreteness of $G_0$.
    \end{infoblock}
\end{frame}

%---------------------------------------------------
\begin{frame}
  \frametitle{Examples of HDP Mixture Models}
  \begin{itemize}[<+->]
  \item If $H$ is a Normal-Inverse Gamma distribution and $f$ is a
    Normal density, we have a hierarchical Infinite Gaussian Mixture.
  \item If $H$ is itself a Dirichlet distribution, and $f$ is a
    Multinomial mass function, we obtain an {\bf
      hierarchical infinite topic model}: $x$ are words, and $\theta_{jk}$
    parameterize multinomial distributions corresponding to ``topics''
    in a document.
  \end{itemize}
\end{frame}

%---------------------------------------------------
\subsection{Non-Exchangeability}
\label{sec:dirichl-proc-mixt}
\begin{frame}
  \frametitle{Exchangeability vs. Dependence}
  \begin{itemize}[<+->]
  \item A limitation of the HDP is that the components in the bottom
    level measures are {\bf exchangeable}, and independent given
    $G_0$.  Can we allow for particular component pairs to have
    correlated weights across contexts? (E.g., topics cooccur across documents)
  \item Second, both the {\it contexts} and the {\it data points} are
    exchangeable.  Can we couple them through covariates?
  \end{itemize}
\end{frame}


%---------------------------------------------------

\section{HDP Hidden Markov Model with Local Transitions}
\label{sec:hdp-hidden-markov}

%---------------------------------------------------
\subsection{Hidden Markov Models}
\begin{frame}
  \frametitle{Hidden Markov Models}
  One way to incorporate temporal dependence in a mixture model is the
  {\bf Hidden Markov Model}.
  \begin{infoblock}{Hidden Markov Model}
    Let $\{f_k\}_{k=1}^K$ be a finite family of density functions and
    $\{\bpi_k\}_{k=0}^K$ a family of $K$-dimensional Multinomial
    distributions.  Define a {\em latent state
      sequence}, $\{z_t\}_{t=1}^T$ and an {\em observation sequence}
    $\{\bx_t\}_{t=1}^T$ such that
    \begin{align}
      \label{eq:12}
      z_t \given z_{t-1} &\sim \pi_{z_{t-1}} \quad t = 1, \dots, T \\
      \bx_t \given z_t &\sim f_{z_t} \qquad t = 1, \dots, T
    \end{align}
    with $z_0$ defined to be 0.  This defines a {\bf Hidden Markov
      Model} (HMM).
  \end{infoblock}
\end{frame}

%---------------------------------------------------
\begin{frame}
  \frametitle{A Bayesian HMM}
  A standard Bayesian formulation for estimating the emission and
  transition distributions, $\{f_k\}$ and
  $\{\bpi_k\}$, where the $f_k$ are members of a parametric family,
  $f_k(\bx) = f(\bx \given \theta_k)$ is to use priors
  \begin{align}
    \label{eq:5}
    \bpi_k &\stackrel{i.i.d}{\sim} \mathrm{Dirichlet}(\alpha\mathbf{1}_K) \\
    \theta_k &\stackrel{i.i.d.}{\sim} f\text{-}\mathrm{Conjugate}(\xi)
  \end{align}
  It is then straightforward to do Gibbs sampling over these variables
  as well as the latent state sequence (which can be sampled jointly
  using a dynamic programming-based message passing algorithm).
\end{frame}

%---------------------------------------------------
\subsection{The Infinite (HDP) Hidden Markov Model}
\begin{frame}
  \frametitle{An Infinite State Generalization}
  We can allow infinitely many states by replacing the Dirichlet prior
  with a Dirichlet Process prior:
  \begin{align}
    \label{eq:5}
    (\bpi_k, \btheta_k) &\stackrel{i.i.d}{\sim} \mathrm{DP}(\alpha G_0)
  \end{align}
  where $\btheta_k$ represents the vector of emission parameters for the states
  reachable from state $k$.
  \pause

  \vspace{0.2in}

  However, we need $G_0$ to be atomic, to ensure that the $\btheta_k$
  contain overlapping values for different $k$.
  \pause

  \vspace{0.2in}
  Solution: Use a hierarchical prior, with $G_0 \sim
    \mathrm{DP}(\gamma H)$.  This is the Infinite or HDP HMM
    \cite{beal2001infinite,teh2006hierarchical}
\end{frame}

%---------------------------------------------------
\subsection{Adding the Concept of ``Local'' Transitions}
\begin{frame}
  \frametitle{Excessive Exchangeability}
  \begin{itemize}[<+->]
  \item Two properties of HDP-HMM not shared with non-temporal HDP:
    \begin{enumerate}
    \item Contexts (except the first) are random
    \item Set of contexts is identified with set of states
    \end{enumerate}
  \item Self-transitions are not special
  \item No notion of a state topology: $\pi_{kk'}$ should (perhaps) be
    similar (but not identical) to $\pi_{k'k}$; states with similar incoming distributions should (perhaps)
    have similar (but not identical) outgoing distributions.
  \end{itemize}
\end{frame}

%---------------------------------------------------
\begin{frame}
  \frametitle{Making Self-Transitions Special: Two Approaches}
  \begin{itemize}[<+->]
  \item Two solutions to making self-transitions special are:
    \begin{enumerate}
      \item the Sticky HDP-HMM \cite{fox2008hdp}:
        \begin{equation}
          \label{eq:13}
          \bpi_k \sim \mathrm{DP}(\alpha G_0 + \kappa \delta_{\theta_k})
        \end{equation}
      \item the HDP-HSMM \cite{johnson2013bayesian}: rule out
        self-transitions, and model durations separately
      \end{enumerate}
    \end{itemize}
\end{frame}

%---------------------------------------------------
\begin{frame}
  \frametitle{Local Transitions: The HDP-HMM-LT}
  \begin{itemize}[<+->]
  \item I incorporate the notion of state similarity by defining an
    HDP-HMM that favors ``local'' transitions: The HDP-HMM-LT.
  \item Key idea: latent states are located an abstract space 
    on which a symmetric similarity kernel, $\phi$ is defined:
    \begin{align}
      0 \leq \phi(\ell_k, \ell_{k'}) = \phi(\ell_{k'}, \ell_k) \leq \phi(\ell_k, \ell_k) \equiv 1
    \end{align}
  \item The transition probabilities generated by the HDP prior are
    scaled by the corresponding $\phi_{kk'}$.
  \end{itemize}
\end{frame}

%---------------------------------------------------
\begin{frame}
  \frametitle{The HDP-HMM-LT}
  \begin{infoblock}{Definition: HDP-HMM-LT}
    Assume we have a sequence of location pairs $\{(\theta_k, \ell_k)\}$ from some
    distribution, and a similarity kernel $\phi$.  We define
    \begin{align}
      \bbeta &\sim \mathrm{Stick}(\gamma) \\
      \tilde{\bpi}_k &\sim \mathrm{DP}(\alpha\bbeta) \\
      a_{kk'} &= \tilde{\pi}_{k}(k')\phi(\ell_k, \ell_{k'}) \\
      z_t \given z_{t-1} &\sim \sum_{k} \frac{a_{z_{t-1}k}}{\sum_{k'}
        a_{z_{t-1}k'}} \delta_{k} \\
      x_t \given z_t &\sim F(\theta_{z_t})
    \end{align}
  \end{infoblock}
  Note that the normalization term is finite and positive almost surely, since it
  is bounded above by $\sum_{k'} \tilde{\pi}_{k}(k') = 1$, and below by $\tilde{\pi}_{k}(k)$.
\end{frame}

%---------------------------------------------------
\begin{frame}
  \frametitle{A Gamma Process representation}
  By the Gamma Process representation of the DP, we obtain the same
  model by drawing $\bbeta$ as above and setting
  \begin{equation}
    \label{eq:14}
    \tilde{\pi}_{k}(k') = \frac{\pi_{kk'}}{\sum_{k''} \pi_{kk''}}
  \end{equation}
  where
  \begin{equation}
    \label{eq:15}
    \{\pi_{kk'}\}_{k} \stackrel{i.i.d.}{\sim} \Gamm{\alpha
      \beta_{k'}}{1} \qquad k' \geq 1
  \end{equation}
  It is known that the L\'evy measure underlying the Gamma process
  meets the sufficient conditions for the normalization constant to be
  positive and finite almost surely, namely
  \begin{equation}
    \label{eq:16}
    \int_{\mathbb{R}^+}\rho(d\pi) = +\infty \qquad \int_{\mathbb{R}^{+}}
    (1 - e^{-\pi}) \rho(d\pi) < \infty
  \end{equation}
\end{frame}

%---------------------------------------------------
\begin{frame}
  \frametitle{A Gamma Process representation}
  Combining the two normalizations yields
  \begin{infoblock}{HDP-HMM-LT (Gamma Process Representation)}
    \begin{align}
      \bbeta &\sim \mathrm{Stick}(\gamma) \\
      \pi_{kk'} &\sim \Gamm{\alpha\beta_{k'}}{1} \\
      a_{kk'} &= \pi_{kk'}\phi(\ell_k, \ell_{k'}) \\
      z_t \given z_{t-1} &\sim \sum_{k} \frac{a_{z_{t-1}k}}{\sum_{k'}
        a_{z_{t-1}k'}} \delta_{k} \\
      x_t \given z_t &\sim F(\theta_{z_t})
    \end{align}
  \end{infoblock}
\end{frame}
%---------------------------------------------------
\begin{frame}
  \frametitle{Loss of Conjugacy for $\bpi$}
  A difficulty introduced by the rescaling is that the likelihood for
  $\pi$, fixing the sequence $\bz$ is no longer conjugate, due to the
  normalization.  Let
  $n_{kk'}$ be the number of transitions from $k$ to $k'$ in $\bz$.
  Then
  \begin{equation}
    \label{eq:17}
    p(\bz \given \bpi, \bphi) = \prod_{k} \left(\sum_{k''}
      \pi_{kk'}\phi_{kk'}\right)^{-n_{k\cdot}} \prod_{k'} (\pi_{kk'}\phi_{kk'})^{n_{kk'}}
  \end{equation}
  But we can restore conjugacy by introducing auxiliary variables.
\end{frame}
%---------------------------------------------------
\begin{frame}
  \frametitle{The Markov Process With Failed Jumps Representation}
  A discrete time Markov chain is obtained from a pure jump Markov process by
  allowing self-jumps, and marginalizing time.
  \pause

  \vspace{0.2in}

  Let $A$ be a rate matrix for such a process.  Then time spent in
  state $k$ is distributed $\mathsf{Exp}(\sum_{k'} a_{kk'})$ and is
  independent of the state jumped to, which is distributed by
  the normalized $k$th row of $A$.
\end{frame}
%---------------------------------------------------
\begin{frame}
  \frametitle{The Markov Process With Failed Jumps Representation}
  Given $n_{k\cdot} = \sum_{k'} n_{kk'}$ visits to state $k$, the
  chain will spend
  \begin{equation}
    \label{eq:18}
    u_k \sim \Gamm{n_{k\cdot}}{\sum_{k'}a_{kk'}}
  \end{equation}
  time there.
  
  \pause
  \vspace{0.2in}

  The augmented likelihood is now
  \begin{align}
    \label{eq:21}
    p(\bz, \bu \given \bpi, \bphi) &= \prod_{k}\Gamma(n_{k\cdot})^{-1}
    e^{-u_k \sum_{k'}a_{kk'} }\prod_{k'}
    (\pi_{kk'}\phi_{kk'})^{n_{kk'}} \\
    &= \prod_{k}\Gamma(n_{k\cdot})^{-1} \prod_{k'}
    (\pi_{kk'}\phi_{kk'})^{n_{kk'}} e^{-u_k \phi_{kk'} \pi_{kk'} }
  \end{align}
\end{frame}
% ---------------------------------------------------
\begin{frame}
  \frametitle{Introducing failed jumps}
  Suppose also that while in state $k$, unsuccessful attempts to jump
  to state $k'$ are made at rate $\pi_{kk'} - a_{kk'} = \pi_{kk'}(1 -
  \phi_{kk'})$.  The total number of these is
  \begin{equation}
    \label{eq:19}
    q_{kk'} \sim \Pois{\pi_{kk'}(1 - \phi_{kk'}) u_k}
  \end{equation}
\end{frame}
%---------------------------------------------------
\begin{frame}
  \frametitle{Augmented Likelihood for $\bpi$}
  Augmenting the data with $\bu$ and $\bQ$, the likelihod for $\bpi$
  is now
  \begin{align}
    \label{eq:20}
    \mathcal{L}(\bpi \given \bz, \bu, \bQ; \bphi) &\propto \prod_{k}\prod_{k'}
    \pi_{kk'}^{n_{kk'}} e^{-u_k\phi_{kk'}\pi_{kk'}}
    \\
    &\qquad \times e^{-u_k(1-\phi_{kk'})\pi_{kk'}}
    \frac{\left(u_k(1-\phi_{kk'})\pi_{kk'}\right)^{q_{kk'}}}{q_{kk'}!}
    \\
    &\propto \prod_{k}\prod_{k'}
    \pi_{kk'}^{n_{kk'} + q_{kk'}} e^{-u_k\pi_{kk'}}
  \end{align}
  which is conjugate to the Gamma prior.
\end{frame}
%---------------------------------------------------
\subsection{Posterior Inference}
\begin{frame}
  \frametitle{Completing the Gibbs Sampler}
  \begin{itemize}[<+->]
  \item Conditioned on $A$ and the observations, we can sample the
    state sequence $\bz$ jointly with standard HMM message passing.
  \item We can sample $\bpi$ and its hyperparameters ($\bbeta$,
    $\alpha$, and $\gamma$) jointly as well by the factorization
    \begin{equation}
      \label{eq:22}
      p(\gamma, \alpha, \bbeta, \bpi \given \mathcal{D}) = p(\gamma
      \given \mathcal{D}) p(\alpha \given \mathcal{D}) p(\bbeta \given
      \gamma, \mathcal{D}) p(\bpi \given \bbeta, \alpha, \mathcal{D})
    \end{equation}
    where $\mathcal{D}$ is the augmented ``data''.
  \item These terms require marginal likelihoods for $\bbeta$,
    $\alpha$ and $\gamma$.
  \end{itemize}
\end{frame}
%---------------------------------------------------
\begin{frame}
  \frametitle{Marginal Likelihoods}
  The marginal likelihood for $\bbeta$ (integrating out $\bpi$)
    is
    \begin{align}
      \label{eq:23}
      \mathcal{L}(\bbeta \given \mathcal{D}) &\propto \int \prod_{k}
      \prod_{k'} \Gamma(\alpha\beta_{k'})^{-1} \pi_{kk'}^{\alpha\beta_{k'} + n_{kk'} + q_{kk'} - 1}
      e^{-(1 + u_k)\pi_{kk'}} d\bpi \\
      &\propto \prod_{k} \prod_{k'} (1 + u_k)^{-\alpha\beta_{k'}}
      \frac{\Gamma\left(\alpha\beta_{k'} + n_{kk'} +
          q_{kk'}\right)}{\Gamma(\alpha\beta_{k'})} \\
      &\propto \prod_{k} (1 + u_{k})^{-\alpha} \prod_{k'} \frac{\Gamma\left(\alpha\beta_{k'} + n_{kk'} +
          q_{kk'}\right)}{\Gamma(\alpha\beta_{k'})} \\
      &\propto \prod_{k} \prod_{k'} \sum_{m=1}^{n_{kk'} + q_{kk'}}
      s(n_{kk'} + q_{kk'}, m) \alpha^m \beta_{k'}^m
    \end{align}
\end{frame}
%---------------------------------------------------
\begin{frame}
  \frametitle{Marginal Likelihoods}
  \begin{itemize}
  \item So we can employ the auxiliary variable method of
    \cite{escobar1995bayesian}, introducing a collection of
    random $m_{kk'}$ whose distributions depend on $n_{kk'}$, $q_{kk'}$,
    $\alpha$ and $\bbeta$.
  \item $m_{kk'}$ represents the number ``sticks''
    assigned to component $k'$ in context $k$, during the
    DPs generating $\bpi$.
  \item After adding $\bM$ to $\mathcal{D}$, the $\bbeta$ likelihood is conjugate to
    the DP prior.
\end{itemize}
\end{frame}
%---------------------------------------------------
\begin{frame}
  \frametitle{Marginal Likelihoods}
  With the addition of $\bM$, the marginal likelihood for $\alpha$ is
  now simply
  \begin{align}
    \label{eq:24}
    \mathcal{L}(\alpha \given \mathcal{D}) &\propto \prod_{k} (1 +
    u_k)^{-\alpha} \prod_{k'} \alpha^{m_{kk'}} \\
    &\propto \alpha^{m_{\cdot\cdot}} e^{-\sum_k \log(1 + u_k) \alpha}
  \end{align}
  which is conjugate to a Gamma prior.  This is simpler than in the
  formulation of either \cite{escobar1995bayesian} or
  \cite{teh2006hierarchical} due to the presence of $\bu$.
\end{frame}

%---------------------------------------------------
\begin{frame}
  \frametitle{Marginal Likelihoods}
  We can further integrate out $\bbeta$ to obtain a marginal likelihood
  for $\gamma$.  Collapsing all unrepresented components of $\bbeta$
  into a single component $\beta_{*}$ yields a degenerate Dirichlet prior, and so
    \begin{align}
      \label{eq:23}
      \mathcal{L}(\gamma \given \mathcal{D}) &\propto \int
      \beta_{*}^{\gamma} \prod_{k} \beta_k^{m_{\cdot k}} d\bbeta \\
      &\propto \frac{\Gamma(\gamma) \prod_{k} \Gamma(m_{\cdot
          k})}{\Gamma(\gamma + m_{\cdot\cdot})} \\
      &\propto \int_{0}^{1} t^{\gamma - 1} (1 - t)^{m_{\cdot\cdot} -
        1} dt
    \end{align}
    and so adding one more auxiliary variable $t \sim
    \Beta{\gamma}{m_{\cdot\cdot}}$ yields a marginal likelihood for
    $\gamma$ which is conjugate to a Gamma prior.
\end{frame}
%---------------------------------------------------
\begin{frame}
  \begin{infoblock}{Interim Summary of Contributions}
    \begin{itemize}
      \item I have introduced a new model, the HDP-HMM-LT, in which there is a
      notion of topology on the transition state space, such that
      transitions are more likely between nearby states.
      \item I have formulated the HDP-HMM-LT as the marginalization of
        another process, the Markov Process With Failed Jumps.
      \item By ``reinstating'' selected functions of the marginalized
        variables, the model admits a straightforward Gibbs sampler
      \item The HDP parameters, $\bpi$, $\bbeta$, $\alpha$ and
        $\gamma$ can be Gibbs-sampled jointly, conditioned on the
        augmented data.
    \end{itemize}
\end{infoblock}
\end{frame}


\section{Two DP Models for Context-Sensitive Phrase Structure Grammar}
\label{sec:two-dp-models}
\subsection{Context-Free Grammars}
% ---------------------------------------------------
\begin{frame}
  \frametitle{Constituency Trees}
  \Tree [.S 
            [.NP [.JJ Natural ] [.NN language ] [.NN processing ] ]
            [.VP [.VBZ is ] [.JJ cool ] ] ]
\end{frame}

\begin{frame}
  \frametitle{Context-Free Grammar}
  Ordinary Context Free Grammar: syntactic
    constituents (nonterminal phrases) expanded into sequences of smaller phrase
    nodes and ``preterminals'' (part of speech tags).
  \begin{exblock}{}
  \begin{tabular}{lll}
    $S \to NP\ VP$ & $NP \to JJ\ NN\ NN$ & $VP \to VBZ\ JJ$ \\
    $JJ \to \text{Natural}$ & $NN \to \text{language}$ & $NN \to \text{processing}$ \\
    $V \to \text{is}$ & $JJ \to \text{cool}$ & \\
  \end{tabular}
\end{exblock}
\end{frame}


% ---------------------------------------------------
\begin{frame}
  \frametitle{Now with Probabilities}
  \begin{itemize}
  \item But the same LHS can expand in more than one way.
  \item {\bf Probabilistic} Context-Free Grammar associates each
    production with a conditional probability of the RHS given the LHS.
  \end{itemize}
\begin{exblock}{}
  \begin{tabular}{lll}
    $S \stackrel{0.7}{\to} NP\ VP$ & $NP \stackrel{0.02}{\to} JJ\ NN\ NN$ & $VP \stackrel{0.2}{\to} V\ JJ$ \\
    $JJ \stackrel{0.002}{\to} \text{Natural}$ & $NN \stackrel{0.001}{\to} \text{language}$ & $NN \stackrel{0.0001}{\to}$
    \text{processing} \\
    $V \stackrel{0.01}{\to} \text{is}$ & $JJ \stackrel{0.002}{\to}
    \text{cool}$ & 
  \end{tabular}
\end{exblock}
\pause
  \begin{itemize}
  \item Context-free assumption: each sequence is conditionally independent
    of everything else given the parent.
  \item Therefore, tree probability is product of production probabilities.
  \end{itemize}
\end{frame}
% ---------------------------------------------------
\begin{frame}
  \frametitle{Problems With This Structure}
  \begin{itemize}[<+->]
  \item This grammar is both too flexible and too inflexible.
  \item Too inflexible: Context-free assumption is unrealistic;
    grandparents, aunts, etc. matter.  Too few parameters!
  \item Too flexible: Each sequence (say, $JJ\ NN\ NN$) treated as a
    distinct category, regardless of similarity (to, say, $NN\ NN$).  Too many parameters!
  \end{itemize}
\end{frame}

% ---------------------------------------------------
\begin{frame}
\frametitle{Lexicalizing and Factoring the Production Probabilities}
\begin{itemize}[<+->]
\item \cite{collins2003head} augments representation by
  labeling constituent nodes with lexical and ``head/modifier''
  information.
\item Each constituent has a single head word.
\item Multiple production types
\item Modifiers generated one at a time (conditionally independent of
  each other given parent and head).
\end{itemize}
\end{frame}
% ---------------------------------------------------

\begin{frame}
  \frametitle{Head-Driven Lexicalized Grammar}
\begin{columns}
  \begin{column}{0.5\textwidth}
\begin{figure}[t]

  \vspace{-2.5in}

{\tiny
  \Tree [.S(VBZ,is)
            [.NP(NN,processing) 
                [.NPB(NN,language) 
                    [.JJ natural ]
                    [.*NN language ] ]
                [.*NPB(NN,processing)
                    [.*NN processing ] ] ]
            [.*VP(VBZ,is)
                 [.*VBZ is ]
                 [.JJ cool ] ] ] 
}
\end{figure}
\end{column}
\begin{column}{0.5\textwidth}

\vspace{0.7in}

{\footnotesize
\begin{align*}
  ROOT &\stackrel{\text{head}}{\to} S(VBZ,\text{is}) \\
  S(VBZ,\text{is}) &\stackrel{\text{head}}{\to} VP(VBZ,\text{is}) \\
  VP(VBZ,\text{is}) &\stackrel{\text{head}}{\to} VBZ(\text{is}) \\
  VP(VBZ,\text{is}) & \stackrel{\text{right}}{\to} JJ(\text{cool}) \\ 
  S(VBZ,\text{is}) &\stackrel{\text{left}}{\to} NP(NN,\text{processing}) \\
  NP(NN,\text{processing}) & \stackrel{\text{head}}{\to}
  NBP(NN,\text{processing})
  \\
  NBP(NN,\text{processing}) & \stackrel{\text{head}}{\to}
  NN(\text{processing}) \\
  NP(NN,\text{processing}) & \stackrel{\text{left}}{\to}
  NPB(NN,\text{language}) \\
  NPB(NN,\text{language}) & \stackrel{\text{head}}{\to} NN(\text{language})\\
  NPB(NN,\text{language}) & \stackrel{\text{left}}{\to} JJ(\text{natural}) \\
\end{align*}
}
\end{column}
\end{columns}
\end{frame}

% ---------------------------------------------------
\begin{frame}
  \frametitle{Parameter Estimation}
  \begin{itemize}
  \item For modifiers, condition on (sister) head, as well as parent.
  \item But contexts now contain words --- very little training data per
    context!
  \item Solution: ``smoothing'' distributions for similar contexts
  \end{itemize}
\pause
{\footnotesize
\begin{align*}
  &S(JJ,cool) \stackrel{\text{head}}{\to} VP\ 
  \vert\ \text{Parent} = \mathrm{Root}
\end{align*}
\begin{align*}
&P(VP\ \vert\ \mathrm{Root}, S, JJ, \text{cool}) \\
  & = \lambda_0 \hat{P}(VP\ \vert\ \mathrm{Root}, S, JJ, \text{cool})
  \\ & \qquad + (1 - \lambda_0)\big(\lambda_1 \hat{P}(VP\ \vert\ \mathrm{Root}, S, JJ) \\
    &\qquad \qquad \qquad + (1 - \lambda_1)(\lambda_2 \hat{P}(VP\ \vert\ \mathrm{Root}) + (1-\lambda_2) \varepsilon)\big)
\end{align*}
}
\end{frame}

% ---------------------------------------------------
\begin{frame}
  \frametitle{How Should We Choose $\lambda$s?}
  Collins sets smoothing parameters from the data as
    \begin{equation*}
      \label{eq:2}
      \lambda_{A} = \frac{c_A}{c_A + b u_A}
    \end{equation*}
    \begin{center}
    \begin{tabular}{l}
      $c_A$: \# of observations of context $A$ \\
      $u_A$: \# of distinct outcomes in context $A$ (the
      ``diversity'') \\
      $b$: tunable parameter shared over all contexts
    \end{tabular}
  \end{center}
\end{frame}
% ---------------------------------------------------
\begin{frame}
  \frametitle{Intuition}
  \begin{itemize}[<+->]
  \item More data for context (high $c_A$) $\to$ trust the MLE more
  \item More distinct outcomes (high $u_A$) $\to$ high entropy $\to$
    need more data to get a reliable estimate
  \end{itemize}
\end{frame}

\subsection{An HDP Formulation of Backoff Smoothing}
\label{sec:an-hdp-formulation}
% ---------------------------------------------------
\begin{frame}
  \frametitle{A Bayesian Interpretation of Smoothing}
  \begin{itemize}[<+->]
  \item With only one smoothing level, i.e.,
    \begin{equation*}
      \label{eq:1}
      P_A = \lambda \hat{P}_A + (1 - \lambda) \varepsilon
    \end{equation*}
    this would be the predictive distribution assuming a
    Dirichlet-multinomial model, with a symmetric prior over $1 /
    \varepsilon$ categories and concentration hyperparameter
    related to $(1 - \lambda)$.  Specifically
    \begin{equation}
      \label{eq:25}
      \lambda = \frac{c_A}{c_A + \alpha}
    \end{equation}
  \end{itemize}
\end{frame}

% ---------------------------------------------------
\begin{frame}
  \frametitle{A Bayesian Interpretation of Smoothing}
  We can extend an HDP to arbitrary levels by letting $\pi_K(A)$
  represent the predictive distribution for context $A$ at level $K$ (where
  smaller $K$ collapse prefix-equivalent $A$s).  Then
    \begin{equation*}
      \label{eq:1}
      \pi_K(A) = \lambda_K \hat{\pi}_K(A) + (1 - \lambda_k) \pi_{K-1}(A)
    \end{equation*}
    where $\hat{\pi}$ is the empirical distribution,
    \begin{equation}
      \label{eq:26}
      \lambda_K = \frac{m_K(A)}{m_K(A) + \alpha_K}
    \end{equation}
    and $m_K(A)$ is the total number of distinct masses (``sticks'') with
    which the observations in context $A$ are associated, and
    $\alpha_K$ is the concentration parameter of the DP at level $K$.
\end{frame}

% ---------------------------------------------------
\begin{frame}
  \frametitle{Estimating $\alpha$}
  The set of $m_K(A)$ for all the contexts at level $K$ is
  probabilistically dependent on $\alpha_{K}$, since higher $\alpha_K$
  leads to more a more entropic set of stick weights.
  \cite{antoniak1974mixtures} showed that, for a fixed number $n$ of draws
  from the distribution, the number of distinct ``sticks'' represented
  has distribution given by
  \begin{equation}
    \label{eq:27}
    p(m \given \alpha, n) = \frac{\Gamma(\alpha)}{\Gamma(\alpha +
      n)} s(n,m) \alpha^m
  \end{equation}
  This is proportional (in $\alpha$) to
  \begin{equation}
    \label{eq:28}
    \alpha^m \int_0^1 w^{\alpha-1}(1-w)^{n-1} dw
  \end{equation}
  so if we draw $w \sim \Beta{\alpha}{n}$, the augmented likelihood is
  conjugate to a Gamma prior with shape and rate updates $m$ and
  $-\log(w)$, respectively.
\end{frame}
% ---------------------------------------------------
\begin{frame}
  \frametitle{Estimating $\alpha$}
  For fixed $w$, the posterior mean for $\alpha$ is asymptotically
  proportional to $m$.  Hence if we approximate $m$ by the diversity $u$ and all
  $w$s by a fixed constant $w_0$ (perhaps a function of the sample size), we
  might estimate $\alpha$ by $w_0 u$, as Collins does:
    \begin{equation*}
      \label{eq:2}
      \lambda_{A} = \frac{c_A}{c_A + b u_A}
    \end{equation*}
    \begin{center}
    \begin{tabular}{l}
      $c_A$: \# of observations of context $A$ \\
      $u_A$: \# of distinct outcomes in context $A$ (the
      ``diversity'') \\
      $b$: tunable parameter shared over all contexts
    \end{tabular}
  \end{center}
\end{frame}

% ---------------------------------------------------
\begin{frame}
  \frametitle{A true HDP model}
  \begin{itemize}[<+->]
  \item Instead of these heuristic approximations, I propose Gibbs sampling
    the $m$,
    $w$ and $\alpha$ to derive a Monte Carlo estimate of the predictive
    probabilities.
  \item This can perhaps improve parsers trained on annotated corpora
    (empirical study to come).
  \item Further, by specifying a full probabilistic model, we can account for missing or
    incomplete context data in the training bank of parse trees, such
    as semantic information (informed by paired images).
\end{itemize}
\end{frame}



% ---------------------------------------------------
\begin{frame}
  \frametitle{Conditioning Grammar on Semantics}
We want the probability of this...
 {\tiny
  \Tree [.S(VBZ,is)
            [.NP(NN,lamp) 
                [\qroof{a white lamp}.*NPB ]
                [.PP(IN,on) [.*IN on ] [\qroof{a table}.NPB ] ] ]
            [.*VP(VBZ,is)
                 [.*VBZ is ]
                 [.PP(IN,in)
                     [.*IN in ]
                     [.NP(NN,front) 
                         [.*NPB(NN,front) [.NN front ] ]
                         [.PP(IN,of) 
                             [.*IN of ]
                             [.NPB(NN,bed) [.det a ] [.JJ green ] [.*NN bed ] ] ] ] ] ] ]
}
\end{frame}

\begin{frame}
to depend on this:
\begin{figure}[t]
 {\tiny
  \Tree [.{\sc Focus}(lamp)
            [.T:{\sc front-of}(lamp,bed) 
                H:front-of
                [.T:{\sc on}(lamp,table) 
                    H:on
                    [.T:lamp(white) H:lamp A:white ] 
                    [.B:table() H:table A:NULL ] ]
                [.B:bed(green) H:bed A:green ] ] ]
}
\end{figure}
\end{frame}


% ---------------------------------------------------
\begin{frame}
  \frametitle{Sparse Dependency Assumption}
    \begin{itemize}[<+->]
    \item That's a lot of structure to condition on.  Smoothing only
      helps so much.
    \item Intuition: A given syntactic subtree expresses a particular
      ``semantic constituent''.
      \begin{align*}
        front(lamp, bed) &\leftarrow S(VBZ, is) \\
        bed(green) &\leftarrow NP(NN,bed)
      \end{align*}
    \item Moreover, correspondences are largely ``continuous''
      (connected graphs map to connected graphs).
    \item Hence, add semantic features to the head/modifier features,
      and define new production types.
    \end{itemize}
\end{frame}

% ---------------------------------------------------
\begin{frame}
  \frametitle{Semantic Features and Events in $P(T \vert \psi, \mathcal{G})$}
  \begin{itemize}
  \item Syntactic root is associated with root predicate; each step down
    syntactic tree associated with (a) no move, (b) move down, or (c)
    move to null, in semantic tree.
  \end{itemize}
\begin{figure}[t]
  \vspace{-0.4in}

  \hspace{-0.1in}

 {\tiny
  \Tree 
  [.*VP(VBZ,is,{\em front(lamp,bed)})
      [.*VBZ({\em front}) is ]
          [.PP(IN,in,{\em front(lamp,bed)})
              [.*IN({\em front}) in ]
                  [.NP(NN,front,{\em front(lamp,bed)})
                      [.*NPB(NN,front,{\em front}) 
                          [.NN({\em front}) front ] ] 
                          [.PP(IN,of,{\em front(lamp,bed)})
                              [.*IN({\em front}) of ]
                              [.NPB(NN,bed,{\em bed(green)}) 
                                  [.det($\emptyset$) a ] 
                                  [.JJ({\em green}) green ] 
                                  [.*NN({\em bed}) bed ] ] ] ] ] ]
}
\end{figure}
\end{frame}

% ---------------------------------------------------
\begin{frame}
  \frametitle{Semantic Features and Events in $P(T \vert \psi, \mathcal{G})$}
  \begin{itemize}
  \item Simply adds new production types to the grammar.
  \item For a fixed mapping, we can employ the same context-sensitive
    grammar model to estimate the likelihood of a parse.
  \item Use this as a likelihood to sample (a) mappings, and (b) the
    semantic trees themselves
  \end{itemize}
\end{frame}

\subsection{Future Work: A Similarity-Based Alternative}
\label{sec:simil-based-altern}
\begin{frame}
  \frametitle{A Similarity-Based Alternative}
  \begin{itemize}
  \item The HDP model requires context features to be ordered in
    advance.
  \item No tying between contexts that differ on a feature in the
    first ``tier''.
  \item Instead: couple context distributions via an overall
    similarity measure.
  \end{itemize}
\end{frame}

\begin{frame}
  \frametitle{A Similarity-Based Alternative}
  \begin{itemize}[<+->]
  \item Treat each context distribution, $P_A$ as an infinite mixture
    of principal ``topic'' distributions
    \begin{equation}
      \label{eq:29}
      P_A = \sum_{k=1}^\infty w_k(A) P'_k, \qquad \sum_k w_k(A) = 1
    \end{equation}
  \item Introduce random topic indicators, $\{z_i\}$ for each observed
    production, with 
    \begin{align}
      z_i \given A_i &\sim \sum_{k=1}^{\infty} w_k(A_i) \delta_k \\
      x_i \given z_i &\sim P_{z_i}
    \end{align}
  \item Key: coupled prior on $\{w(A)\}_A$ so that $w(A)$ and $w(A')$
    are similar when $A$ and $A'$ are similar.
\end{itemize}
\end{frame}

\begin{frame}
  \frametitle{The Distance-Dependent Chinese Restaurant Process}
  \begin{itemize}[<+->]
  \item Several approaches exist to place a distribution on partitions
    that bias groupings based on similarity on covariates: 
    \cite{maceachern2000dependent,dunson2007bayesian,dahl2008distance,muller2011product}
  \item However, all assume that the covariates are known, which is
    not the case when the context consists of outcomes above in the
    tree.
  \item The Distance-Dependent Chinese Restaurant Process
    \cite{blei2011distance} assumes this as well, but admits a
    straightforward modification to resample context.
  \end{itemize}
\end{frame}

\begin{frame}
  \frametitle{The Chinese Restaurant Process}
  \begin{itemize}[<+->]
  \item By integrating out the ``stick weights'', we get a marginal
    distribution on partitions of observations to ``sticks'' called
    the Polya Urn Process, aka, the {\bf Chinese Restaurant Process}.
  \item Conditioned on a partition of $N$ other observations, into $K$
    clusters (numbered $1, \dots, K$), with $n_k$ in cluster $k$, 
    the cluster assignment $z_{N+1}$ of the
    $N+1$ data point has distribution
    \begin{equation}
      \label{eq:30}
      p(z_{N+1} = k) \propto
      \begin{cases}
        n_k & k = 1, \dots, K \\
        \alpha & k = K+1
      \end{cases}
    \end{equation}
  \end{itemize}
\end{frame}

\begin{frame}
  \frametitle{The Distance-Dependent CRP}
  \begin{itemize}[<+->]
  \item Equivalently, assign observation $N+1$ to the same stick as
    another observation with probability proportional to $N$ and to a
    new stick with probability proportional to $\alpha$.  Conditioned
    on the first case, choose an observation uniformly.
  \item The {\bf Distance-Dependent CRP} generalizes this uniformity.
    Given a similarity function $\phi(\bx, \bx')$, where $\bx$ and
    $\bx'$ are covariates, link observation $i$ to
    observation $j$ according to
    \begin{equation}
      \label{eq:31}
      p(c_i = j) \propto
      \begin{cases}
        \phi(\bx_i,\bx_j) & i \neq j \\
        \alpha & i = j
      \end{cases}
    \end{equation}
  \item The self-link case corresponds to creating a new cluster.
  \end{itemize}
\end{frame}

\begin{frame}
  \frametitle{The Distance-Dependent CRP for Context-Sensitive Grammars}
  \begin{itemize}[<+->]
  \item We can then alternate sampling the per-cluster outcome
    variables (which in our case, unlike in the original ddCRP, needs
    to take into account the links, since it will change some
    contexts and hence the link probabilities), 
    and sampling the links.
  \item Moreover, we can sample parameters of the similarity function
    conditional on the links to do relevance-determination.
  \end{itemize}
\end{frame}

% % ---------------------------------------------------
% \begin{frame}
%   \frametitle{Semantic Features and Events in $P(T \vert \psi,
%     \mathcal{G})$}
% \vspace{-0.25in}
%   \begin{itemize}
%   \item Productions yielding constituent labels depend on a
%     hierarchy of context ``history''.  Define semantic ``move events'' analogously.
%   \end{itemize}
% {\scriptsize
% \begin{center}
% \vspace{-0.2in}
% \begin{tabular}[h]{|c|c|c|c|c|c|} \hline
%   \multirow{2}{*}{{\bf type}} & \multirow{2}{*}{{\bf generated}} & \multicolumn{4}{|c|}{{\bf conditioned on}} \\ \cline{3-6}
%   & & Level 0 & Level 1 & Level 2 & Level 3 \\ \hline
%   {\tt r1} & ch, th & {\bf sh}, {\bf cp} = {\tt TOP} & {\bf sha} & & \\ \hline
%   {\tt r2} & wh & th, {\bf sh} & ch & cp = TOP, {\bf sha} & \\ \hline
%   {\tt hsem} & {\bf sh}, {\bf sha} & {\bf sp}, cp & th, {\bf spa} & wh & \\ \hline
%   {\tt h} & ch & {\bf sh}, cp & th, {\bf sha} & wh & \\ \hline
%   {\tt msem} & {\bf sm}, {\bf sma} & {\bf sp}, cp & {\bf sh}, ch & th, {\bf spa}, {\bf sha} & wh \\ \hline
%   {\tt m1} & cm, tm & cp, {\bf sm} & ch, {\bf sp}, {\bf sh}, side & th, {\bf sma}, {\bf spa}, {\bf sha} & wh \\ \hline
%   {\tt m2} & wm & tm, {\bf sm} & cm, cp, ch, {\bf sp}, {\bf sh}, side & th, {\bf sma}, {\bf spa}, {\bf sha} & wh \\ \hline
% \end{tabular}
% \end{center}
% \begin{tabular}{ll}
% {\tt r1,r2: root events} & {\tt cp = constituent of parent}\\
% {\tt h: syntactic head event} & {\tt ch = constituent of (sister) head} \\
% {\tt m1,m2: syntactic modifier events} & {\tt cm = constituent of modifier (self)}\\
% {\tt hsem: semantic event at head} & {\tt w* = word of *} \\
% {\tt msem: semantic event at modifier} & {\tt t* = PoS tag of *}\\
% & {\tt s* = (head) semantic label of *} \\
% & {\tt s*a = arguments of semantic label of *}
% \end{tabular}
% }
% \end{frame}


% % ---------------------------------------------------
% \section{Training}
% \begin{frame}
%   \frametitle{Learning $\mathcal{G}$}
%   \begin{itemize}
%   \item Grammar parameters in $\mathcal{G}$ consist of conditional
%     probabilities of two types of syntactic productions (head and
%     modifier), as well as new semantic
%     ``productions'' (type of step taken on the semantic tree)
%   \item All production events are conditioned on syntactic and semantic
%     ``history''.
%   \item Estimate parameters from human-verified 'ground truth' parse
%     trees, associated with hand-annotated scenes.
%   \item Problem: no semantic ground truth, so learning is only semi-supervised.
%   \end{itemize}
% \end{frame}

% % ---------------------------------------------------
% \begin{frame}
% \frametitle{MCMC Algorithm for Inferring $\Psi$ for Training Data}
% \begin{itemize}
% \item Given $\Psi$ and augmented $T$s ($T^\prime$), it is straightforward to set parameters of
%   $\mathcal{G}$ (just counting).
% \item Use prior on $\Psi$ given ground truth $Z$, and likelihood of
%   $T$ given $\Psi$, and do MCMC.
%   \begin{enumerate}
%   \item Initialize $\Psi$ from prior.
%   \item Initialize $T^\prime$ deterministically given $\Psi$.
%   \item Gibbs sampling: Sample $T^\prime$ for one sentence at a time, given
%     others and $\Psi$.
%   \item Then sample $\Psi$ given $T^\prime$.
%   \end{enumerate}
% \end{itemize}
% \end{frame}

% % ---------------------------------------------------
% \section{Inference}
% \begin{frame}
%   \frametitle{Inference}
%   \begin{figure}[h]
%     % \centering
%     \includegraphics[width = 1\textwidth]{../img/GenerativeModelAfterTraining.pdf}
%     \caption{Bayes net representation of the probabilistic model after
%       training.}
%     \label{fig:graphical-model}
%   \end{figure}

%     \vspace{-0.35in}


%   \begin{itemize}
%   \item After training, we treat $\mathcal{G}$ as known (for now)
%   \item Goal: Infer posterior, $p(T, \Psi, Z, C \vert S, I,
%     \mathcal{G}) \propto p(Z) p(C) p(I \vert Z, C) p(\Psi \vert Z) p(T
%     \vert \Psi, \mathcal{G}) \mathbbm{1}(S \equiv T)$ using MCMC.
%   \end{itemize}
% \end{frame}

% % ---------------------------------------------------
% \begin{frame}
%   Results in progress...
%   {\footnotesize \bibliography{../bib/dissertation.bib}}
% \end{frame}

% \bibliographystyle{apalike}
\bibliography{../../../manuscripts/sac/bib/scenes_and_captions}

\end{document}
