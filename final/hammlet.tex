In this chapter, 
I describe a generalization of the Hierarchical Dirichlet Process
Hidden Markov Model  \citet{teh2006hierarchical} discussed in Chapter \ref{chapter:HMM-NPBayes}
which introduces a notion of latent similarity between pairs of hidden
states, such that transitions are a priori more likely to occur between states
that are deemed ``similar''.  This is achieved by
placing a kernel function on the space of state parameters which
returns for each pair of states a value between 0 and 1, representing
the degree to which they are similar (in whatever sense is desired for
the application at hand), and
scaling HDP-generated transition probabilities by the similarity between states.  
I will refer to this model as the Hierarchical Dirichlet Process Hidden Markov Model with Local
Transitions (HDP-HMM-LT, or HaMMLeT).  Although this achieves the goal of selectively increasing the
probability of transitions between similar states, inference is made
more complicated since, unlike in the ``vanilla'' HDP-HMM, 
the posterior measure over transition
distributions is no longer a Dirichlet Process --- that is, the prior
is no longer conjugate to the state sequence likelihood.  

I will present an alternative representation of this process that facilitates
inference with an auxiliary variable scheme with the following
interpretation: The discrete time chain is recast as a continuous time Markov Process
in which: (1) some jump attempts fail, (2) the probability of success is
proportional to the similarity between the source and destination
states, (3) only successful jumps are observed,
and (4) the time elapsed between jumps, as well as the number of
unsuccessful jump attempts, are latent variables that are sampled
during MCMC inference.  By introducing these auxiliary latent
variables, nearly all
conditional distributions in the model are members of an exponential
family, admitting exact Gibbs sampling.  The only exception
is the set of similarities, which are defined in an
application-specific way, and require application-specific inference methods.
I present results for a few different choices in the experiments in
Chs \ref{chapter:cocktail-party} through \ref{chapter:music}.

The motivating domain for this model is natural language text, in
which sentences in a document are associated with a set of ``topics'',
the topic sets in successive sentences have a high
degree of overlap, even when they are not identical (so that
neighboring topic vectors are similar), and in which there may be a
high degree of correlation between topics, so that modeling the entry
and disappearance of each topic independently is undesirable.  
The topic vectors are represented using binary vectors, indicating which
topics are ``active'' in the sentence, and to constrain the dynamics 
governing latent state transitions so that
transitions between similar topic vectors are {\it a priori} more
likely, but where certain topics tend to occur together.  The latter
property makes an ordinary factorial HMM undesireable.  

In the remainder of this chapter, 
I first review the transition dynamics in the ``vanilla'' HDP-HMM so
that it is easier to see how the proposed HDP-HMM-LT differs; I then
define the generative process of the HDP-HMM-LT; finally, I introduce
the ``Markov Jump Process With Failed Transitions'' augmented data
representation, which gives rise to a natural Gibbs sampler for the
HDP-HMM-LT model.

\section{Transition Dynamics in the HDP-HMM}
\label{sec:transition-dynamics}

The conventional HDP-HMM \citep{teh2006hierarchical} is based on a 
Hierarchical Dirichlet Process which is defined as follows:

Each of a countably infinite set of states, indexed by $j$, receives a
parameter vector $\theta_j \in \Theta$, according to base measure $H$.  A top-level
weight distribution, $\beta$, is drawn from a stick-breaking
process with parameter $\gamma > 0$, so that state $j$ has overall
weight $\beta_j$, and an emission distribution which is parameterized by $\theta_j$.
\begin{align}
\theta_j &\stackrel{i.i.d.}{\sim} H \\
\beta &\sim \GEM{\gamma}
\end{align}

The actual transition distribution from state $j$, denoted by $\pi_j$
is then drawn from a Dirichlet Process with concentration parameter
$\alpha$ and base measure $\beta$:
\begin{equation}
  \label{eq:1}
  \pi_j \stackrel{i.i.d}{\sim} \DP{\alpha \beta} \qquad j = 1, 2, \dots
\end{equation}

The hidden state sequence is then generated according to the $\pi_j$.
Let $z_t$ be the index of the chain's state at time $t$.  Then we have
\begin{equation}
  \label{eq:4}
  z_t \given z_{t-1}, \pi_{z_{t-1}} \sim \pi_{z_{t-1}} \qquad t = 1, 2, \dots, T
\end{equation}
where $T$ is the length of the data sequence.

Finally, the emission distribution for state $j$ is a function of
$\theta_j$, so that we have
\begin{equation}
  \label{eq:5}
  y_t \given z_{t}, \theta_{z_t} \sim F(\theta_{z_t})
\end{equation}

A shortcoming of this model is that the generative process does not
take into account the fact that the set of source states is the same
as the set of destination states: that is, that the distribution $\pi_j$
has an element which corresponds to state $j$.  Put another way, there
is no special treatment of the diagonal of the transition matrix, so
that self-transitions are no more likely {\it a priori} than
transitions to any other state.  

The Sticky HDP-HMM \citep{fox2008hdp}
addresses this issue by adding an extra mass of $\kappa$ at location $j$ to the base
measure of the DP that generates $\pi_j$.  That is, they replace
\eqref{eq:1} with
\begin{equation}
  \label{eq:6}
  \pi_j \sim \DP{\alpha\beta + \kappa \delta_j}.
\end{equation}
An alternative model is presented by \cite{johnson2013bayesian}, wherein 
state duration distributions are modeled
separately, and ordinary self-transitions are ruled out.  In both of
these models, auxiliary latent variables are introduced to simplify
conditional posterior distributions and facilitate Gibbs sampling.
However, while both of these models have the useful property that
self-transitions are treated as ``special'', they contain no notion of
similarity for pairs of states that are not identical: 
in both cases, when $j \neq j'$, the prior probability of
transitioning from $j$ to $j'$ depends only on the top-level stick
weight associated with state $j'$, and not on the identity or
parameters of the previous state $j$.

\section{An HDP-HMM With Local Transitions}

The goal of the proposed model 
is to add to the transition model the concept of a transition to
a ``nearby'' state, where nearness of $j$ and $j'$ may be defined
deterministically or stochastically in
terms of the emission parameters, $\theta_j$ and $\theta_{j'}$, or may
be based on a latent state geometry that is a priori independent of
the emission distribution.

In order to accomplish this, we first
consider an alternative construction of the transition distributions,
based on the Normalized Gamma Process representation of the Dirichlet
Process \citep{ferguson1973bayesian}.

\paragraph{Notational Conventions} 
In the definitions and derivations that follow, I will adopt the
convention that variables written with no subscript, such as $\theta$,
$\beta$, and $\pi$, represent the collection of all corresponding
subscripted values: for example, $\theta$ is the vector $(\theta_1, \theta_2,
\dots)$, and $\pi$ is the matrix $(\pi_{jj'})$.  For variables that
represent counts, I will use a dot in the subscript to represent a sum
over corresponding individual counts; for example, $n_{j\cdot}$ is
used to represent $\sum_{j'} n_{jj'}$, and $n_{\cdot\cdot}$ means
$\sum_{j}\sum_{j'} n_{jj'}$.

\subsection{A Normalized Gamma Process representation of the HDP-HMM}
\label{sec:normalized-gamma}

Define a random measure, $\mu = \sum_{j=1}^{\infty} \pi_j \delta_{\theta_j}$, where 
\begin{align}
  \pi_j &\stackrel{ind}{\sim} \Gamm{\omega_j}{1} \label{eq:17}\\
  T &= \sum_{j=1}^{\infty} \pi_j \label{eq:18}\\
  \tilde{\pi}_j &= \frac{\pi_j}{T}   \label{eq:16}\\
  \theta_j &\stackrel{i.i.d}{\sim} H \label{eq:19}
\end{align}
and subject to the constraint that $\sum_{j\geq 1} \omega_j < \infty$,
which ensures that $T < \infty$ almost surely, since
\begin{equation*}
  T \sim \Gamm{\sum_j \omega_j}{1}.
\end{equation*}
As shown by Paisley et al. (2011), 
for fixed $\{\omega_j\}$ and $\{\theta_j\}$, $\mu$ is distributed as a Dirichlet
Process with base measure $\nu = \sum_{j=1}^{\infty} \omega_j \delta_{\theta_j}$.
If we draw $\beta$ from a stick-breaking process and then draw a
series $\{\mu_m\}_{m=1}^M$ of
i.i.d. random measures from the above process, setting $w =
\alpha\beta$ for some $\alpha > 0$, then
this defines a Hierarchical Dirichlet Process.  If, moreover, there is
one $\mu_m$ associated with every state $j$, then we obtain the
HDP-HMM.

We can thus write
\begin{align}
  \beta &\sim \GEM{\gamma}   \label{eq:20} \\
  \theta_j &\stackrel{i.i.d.}{\sim} H \label{eq:21}\\
  \pi_{jj'} &\stackrel{ind}{\sim} \Gamm{\alpha \beta_{j'}}{1} \label{eq:22}\\
  T_j &= \sum_{j'=1}^{\infty} \pi_{jj'} \\
  \tilde{\pi}_{jj'} &= \frac{\pi_{jj'}}{T_j} \label{eq:23},
\end{align}
where $\gamma$ and $\alpha$ are prior concentration hyperparameters
for the two DP levels,
\begin{align}
  \label{eq:50}
  p(z_t \given z_{t-1}, \pi) = \tilde{\pi}_{z_{t-1}z_t}
\end{align}
and the observed data
$\{y_t\}_{t\geq 1}$ is distributed as
\begin{equation}
  \label{eq:24}
  y_t \given z_t \stackrel{ind}{\sim} F(\theta_{z_t})
\end{equation}
for some family, $F$ of probability measures indexed by values of $\theta$.

\subsection{Promoting ``Local'' Transitions}
\label{sec:prom-local-trans}

In the preceding formulation, the transition distributions
$\{\pi_{j}\}_{j=1}^\infty$ are independent
conditioned on the top-level weights, $\beta$.  Our goal is to relax this
assumption, in order to allow for the possibility that there may be
correlations between these distributions.  We achieve this by
introducing a notion of a geometry on the state space; that is, that
some pairs of states are ``near'' to each other, and will thus tend to
have similar transition probabilities associated with them.

We associate with each state a location, $\ell_j \in \mathcal{L}$, and define a
``similarity function'', $\phi: \mathcal{L} \times \mathcal{L} \to
[0,1]$, which returns for any pair of locations a measure of how
``close'' they are.

In order to remain agnostic about the extent to which $\phi$ is
based on the emission parameters, I will use $\theta_j$ to represent
the emission parameters for state $j$, and assume that $\ell_j$
determines $\theta_j$, but that $\phi$ may be based on any part of
$\ell_j$, which may include some, all, or none of the information
contained in $\theta_j$.

I will also use the shorthand $\phi_{jj'}$ to represent 
$\phi(\ell_j, \ell_{j'})$, for the sake of readability; but the reader
should keep in mind that whenever $\phi$ appears in a conditional
distribution, it is a constant if and only if $\ell$ is being
conditioned on.

We can generalize the generative process defined in
\eqref{eq:20}-\eqref{eq:23} as follows:
\begin{align}
  \beta &\sim \GEM{\gamma} \\
  \ell_j &\stackrel{i.i.d}{\sim} H \\
  \pi_{jj'} \given \beta, \theta &\sim \Gamm{\alpha \beta_{j'}}{\phi_{jj'}^{-1}} \\
  T_j &= \sum_{j'=1}^{\infty} \pi_{jj'} \\
  \tilde{\pi}_{jj'} &= \frac{\pi_{jj'}}{T_j} \\
  y_t \given z_t \stackrel{ind}{\sim} F(\theta_{z_t})
\end{align}
where $H$ is now a measure on $\mathcal{L}$.
In this new formuation, the expected value of $\pi_{jj'}$ is
$\alpha\beta_{j'}\phi_{jj'}$.  Since a similarity between one object
and another should not exceed the similarity between an object and
itself, we will assume that $\phi_{jj'} = 1$ if $j = j'$.  We will
also assume that the similarity function is symmetric, so that
$\phi_{jj'}  = \phi_{j'j}$ for all $j,j'$.  As
we will see, either or both of these assumptions could be relaxed if
desired in a particular application, but derivations presented here
will make both.

The above model is equivalent to simply drawing the $\pi_{jj'}$ as in
\eqref{eq:20} and scaling each one by $\phi_{jj'}$ prior to
normalization, since it is a general property of Gamma distributions
that, if $X \sim \Gamm{a}{b}$, then $cX \sim \Gamm{a}{b/c}$.

Unfortunately, this formulation complicates inference significantly,
compared to the ordinary HDP-HMM, as the introduction of non-constant 
rate parameters to the prior on
$\pi$ destroys the conjugacy between $\pi$ and $z$, and worse, the
conditional likelihood function for $\pi$ contains a sum
which renders all entries within a row mutually dependent {\em a posteriori}.

\section{The HDP-HMM-LT as a continuous-time 
Markov Jump Process with ``failed'' jumps}
\label{sec:dist-based-filt}

We can gain stronger intuition, as well as simplify posterior
inference, by re-casting the HDP-HMM-LT described in the last section
as a continuous time Markov Jump Process where some of the attempts to jump
from one state to another fail, and where the failure probability
increases as a function of the ``distance'' between the states.

Let $\phi$ be defined as in the last section, and let 
$\beta$, $\theta$ and $\pi$ be defined as in the Normalized Gamma
Process representation of the ordinary HDP-HMM.  Specifically,
\begin{align}
  \label{eq:beta} \beta &\sim \GEM{\gamma} \\
  \ell_j &\stackrel{i.i.d}{\sim} H \\
  \pi_{jj'} \given \beta &\sim \Gamm{\alpha \beta_{j'}}{1}
\end{align}
Now suppose that when the process is in state $j$, jumps to state
$j'$ are made at rate $\pi_{jj'}$.  This defines a continuous-time
Markov Process where the off-diagonal elements of the transition rate
matrix are the off diagonal elements of the $\pi$ matrix.  In addition,
self-jumps are allowed, and occur with rate $\pi_{jj}$.   

If we only observe the jumps and not the durations between jumps, this is an
ordinary Markov chain, whose transition matrix is obtained by
appropriately normalizing the rows of $\pi$.  
If we do not observe the jumps themselves, but
instead an observation is generated once per jump from a distribution that depends
on the state being jumped to, then we have an ordinary HMM.

We modify this process as follows.  
Suppose that each jump attempt from state $j$ to state $j'$ has a
chance of failing, which is an increasing function of the ``distance''
between the states.  In particular, let the success probability be
$\phi_{jj'}$ (recall that we assumed above that $0 \leq \phi_{jj'}
\leq 1$ for all $j,j'$).  Then, the rate of successful jumps from $j$
to $j'$ is $\pi_{jj'}\phi_{jj'}$, and the corresponding rate of unsuccessful jump
attempts is $\pi_{jj'}(1-\phi_{jj'})$.  To see this, denote by
$N_{jj'}$ the total number of jump attempts to $j'$ in a unit
interval of time spent in state $j$.  Since we are assuming the
process is Markovian, the total number of attempts is $\Pois{\pi_{jj'}}$
distributed.  Conditioned on $N_{jj'}$, $n_{jj'}$ will be successful, where
\begin{equation}
  \label{eq:51}
  n_{jj'} \given N_{jj'} \sim \Binom{N_{jj'}}{\phi_{jj'}}
\end{equation}
It is easy to show (and well known) that the marginal distribution of
$n_{jj'}$ is $\Pois{\pi_{jj'}\phi_{jj'}}$, and the marginal
distribution of $N_{jj'} - n_{jj'}$ is
$\Pois{\pi_{jj'}(1-\phi_{jj'})}$.  The rate of successful jumps
from state $j$ overall is then $T_j := \sum_{j'} \pi_{jj'} \phi_{jj'}$.

Let $t$ index jumps, so that $z_t$ indicates the $t$th state visited
by the process (counting self-jumps as marking a transition to a new
time step).  Given
that the process is in state $j$ at discretized time $t$ (that is,
$z_{t} = j$), it is a standard property of Markov Processes that 
the probability that the destination $z_{t+1}$ of the first successful jump
is independent of the time since the last jump, and the probability
that $z_{t+1} = j'$ is proportional to the rate, $\pi_{jj'}\phi_{jj'}$.  

Let $\tau_{t}$ indicate the time elapsed between the $t-1$th and 
and $t$th successful jump (where we assume that the first
observation occurs when the first successful jump from a ``dummy'' initial
state is made).  We have
\begin{equation}
  \label{eq:52}
  \tau_t \given z_{t-1} \sim \Exp{T_{z_{t-1}}}
\end{equation}
where $\tau_t$ is independent of $z_{t}$.

During this period, there will be some number of unsuccessful attempts to
jump to each state, where the rate of unsuccessful jump attempts from
to state $j'$ is given by $\pi_{z_{t-1}}(1 - \phi_{z_{t-1}j'})$.
Denote by $\tilde{q}_{j't}$ the number of unsuccessful jump attempts
to $j'$ during the interval with duration $\tau_t$.  Then we have
\begin{equation}
  \label{eq:53}
  \tilde{q}_{j't} \given z_{t-1}, \tau_t \sim \Pois{\tau_t \pi_{z_{t-1}j'}(1-\phi_{z_{t-1}j'})}
\end{equation}

Define the following additional variables
\begin{align}
  \label{eq:56}
    \mathcal{T}_j &= \{t \given z_{t-1} = j\} \\
    q_{jj'} &= \sum_{t \in \mathcal{T}_j}\tilde{q}_{j't} \\
    u_j &= \sum_{t \in \mathcal{T}_j} \tau_t.
\end{align}
In addition, let $Q = (q_{jj'})_{j,j' \geq 1}$ be the matrix of unsuccessful
jump attempt counts, and $u = (u_j)_{j \geq 1}$ be the vector whose
$j$th entry is the total time spent in state $j$.

Since each of the $\tau_t$ with $t \in \mathcal{T}_j$ are
i.i.d. $\Exp{T_j}$, and since the sum of $n$ Exponential random
variables with shared scale $\lambda$ has a $\Gamm{n}{\lambda}$
distribution, we have
\begin{equation}
u_j \given z, \pi, \ell \stackrel{ind}{\sim} \Gamm{n_{j\cdot}}{T_j}
\end{equation}
where we define $n_{j\cdot} = \sum_{j'} n_{jj'}$, where $n_{jj'}$ is
the number of successful jumps from state $j$ to $j'$ and $n_{j\cdot}$
is the total number of times that state $j$ is visited.

Moreover, since the $\tilde{q}_{j't}$ with $t \in \mathcal{T}_j$ 
are Poisson distributed, the total number of failed
attempts in the total duration $u_j$ is
\begin{equation}
  \label{eq:60}
  q_{jj'} \stackrel{ind}{\sim} \Pois{u_j\pi_{jj'}(1-\phi_{jj'})}.
\end{equation}

Thus if we marginalize out the individual $\tau_t$ and
$\tilde{q}_{j't}$, we have a joint distribution
over $z$, $u$, and $Q$, conditioned on the transition rate
matrix $\pi$ and the success probability matrix $\phi$, which is
\begin{align}
  \label{eq:54}
  p(z, u, Q \given \pi, \ell) &= \left(\prod_{t=1}^T p(z_{t} \given
    z_{t-1})\right) \prod_{j} p(u_j \given z, \pi, \ell)
  \prod_{j'} p(q_{jj'} \given u_j \pi_{jj'}, \phi_{jj'}) \\
  &= \left(\prod_{t} \frac{\pi_{z_{t-1}z_t}\phi_{z_{t-1}z_t}}{T_{z_{t-1}}}\right) \prod_{j}
  \frac{T_j^{n_{j\cdot}}}{\Gamma(n_{j\cdot})} u_j^{n_{j\cdot} - 1}
  e^{-T_j u_j} \\ &\qquad\qquad\times
  \prod_{j'} e^{-u_j\pi_{jj'}(1-\phi_{jj'})} u_j^{q_{jj'}}
  \pi_{jj'}^{q_{jj'}} (1-\phi_{jj'})^{q_{jj'}} (q_{jj'}!)^{-1} \\
  &= \prod_{j} \Gamma(n_{j\cdot})^{-1} u_j^{n_{j\cdot} + q_{j\cdot}-1}
  \\ &\qquad\qquad \times \prod_{j'}
  \pi_{jj'}^{n_{jj'} + q_{jj'}} \phi_{jj'}^{n_{jj'}}
  (1-\phi_{jj'})^{q_{jj'}} e^{-\pi_{jj'}\phi_{jj'}u_j}
  e^{-\pi_{jj'}(1-\phi_{jj'})u_j} (q_{jj'}!)^{-1} \\
  &\label{eq:joint-likelihood} = \prod_{j} \Gamma(n_{j\cdot})^{-1} u_j^{n_{j\cdot} + q_{j\cdot}-1} \prod_{j'}
  \pi_{jj'}^{n_{jj'} + q_{jj'}} \phi_{jj'}^{n_{jj'}}
  (1-\phi_{jj'})^{q_{jj'}} e^{-\pi_{jj'}u_j} (q_{jj'}!)^{-1}
\end{align}

\subsection{An HDP-HSMM-LT modification}
\label{sec:an-hsmm-modification}

In any Hidden Markov Model, the distribution of the number of time
steps for which a given hidden state persists is by definition
a Geometric distribution, where the ``failure'' parameter is the
relevant entry on the diagonal of the transition matrix.  The HDP-HMM
and the HDP-HMM-LT as defined above are no exception.  Although the
Sticky HDP-HMM \cite{fox2008hdp} and the LT
generalization presented here provide mechanisms for which the
diagonal entries of the transition matrix will tend to have greater
mass than the off-diagonal entries, they do not alter the Markovian
assumption, which implies Geometric durations.

The HDP Hidden Semi-Markov Model (HDP-HSMM;
\citet{johnson2013bayesian}) 
gets around this restriction directly, by treating self-transitions as
fundamentally distinct from all other transitions, and modeling state
persistence durations directly.  Should it be desireable to combine
this property with the general similarity bias of the HDP-HMM-LT, 
it is trivial to modify the LT model to incorporate a separate duration model.
We can simply fix the diagonal
elements of $\pi$ to be zero, and allow $D_t$ observations to be
emitted $i.i.d.$ $F(\theta_{z_t})$ at jump $t$, where
\begin{equation}
  \label{eq:95}
  D_t \given z \stackrel{ind}{\sim} g(\omega_{z_t}) \qquad \omega_j
  \stackrel{i.i.d}{\sim} G
\end{equation}
The likelihood then includes the additional term for the $D_t$, and
the only inference step which is affected is that, instead of sampling
$z$ alone, we sample $z$ and the $D_t$ jointly, by defining
\begin{equation}
  z^*_s = z_{\max\{T \given s \leq \sum_{t=1}^T D_t\}}
\end{equation}
where $s$ ranges over the total number of observations, 
and associating a $y_s$ observation sequence with each $z^*_s$.
Inferences about $\phi$ are not affected, since the diagonal
elements are assumed to be 1 anyway.

In the HDP-HSMM as it is presented in \cite{johnson2013bayesian},
zeroing out the diagonal of the transition matrix to isolate
self-transitions to the separate duration model necessitates
renormalization of the other entries so that the rows of the
transition matrix are probability distributions.  As a result, the
conditional posterior for the matrix is no longer in 
an exponential family.  To deal with this, \citeauthor{johnson2013bayesian}
introduce auxiliary
variables which can be interpreted as the diagonal entries of the 
transition matrix prior to zeroing out, as well as the number of
self-transitions that would have occurred for each state had the
transition matrix diagonal governed self-transitions instead of the
separate duration model.  Conditioned on these auxiliary variables,
conjugacy between the transition matrix prior and likelihood is
restored, and Gibbs sampling is able to proceed with exponential
family updates.

In the LT model, on the other hand, we are already representing the
transition matrix in unnormalized form, having rendered the entries
conditionally independent given the $z$ state sequence and the $u$ holding times,
so we are free to clamp the diagonal entries at zero without
introducing a need for additional auxiliary variables 
in order to achieve semi-Markov dynamics.

\subsection{Summary}
\label{sec:model-summary}

I have defined the following augmented generative model for the
HDP-H(S)MM-LT:
\begin{align}
  \label{eq:96}
  \beta &\sim \GEM{\gamma} \\
  \ell_j &\stackrel{i.i.d}{\sim} H \\
  \pi_{jj'} \given \beta &\sim \Gamm{\alpha \beta_{j'}}{1}
  \\
  z_{t} \given z_{t-1}, \pi, \ell &\sim \sum_{j}
  \left(\frac{\pi_{z_{t-1}j}\phi_{z_{t-1}j}}{\sum_{j'}
    \pi_{z_{t-1}j'}\phi_{z_{t-1}j'}}\right)\delta_j \\
  u_j \given z, \pi, \ell &\stackrel{ind}{\sim}
  \Gamm{n_{j\cdot}}{\sum_{j'} \pi_{jj'}\phi_{jj'}} \\
  q_{jj'} \given u, \pi, \ell &\stackrel{ind}{\sim}
  \Pois{u_j(1 - \phi_{jj'})\pi_{jj'}} \\
  \label{eq:likelihood} y_t \given z, \ell &\sim F(\theta_{z_t})
\end{align}

If we are using the HSMM variant, then we simply fix $\pi_{jj}$ to 0
for each $j$, draw
\begin{align}
  \label{eq:97}
  \omega_j &\stackrel{i.i.d}{\sim} G \\
  D_t \given z &\stackrel{ind}{\sim} g(\omega_{z_t}),
\end{align}
for chosen $G$ and $g$, set
\begin{equation}
  \label{eq:98}
  z^*_s = z_{\max\{T \given s \leq \sum_{t=1}^T D_t\}}
\end{equation}
and replace \eqref{eq:likelihood} with
\begin{equation}
  \label{eq:likelihood-hsmm} \by_s \given z, \theta \sim F(\theta_{z^*_s})
\end{equation}

\section{MCMC Inference in the ``Failed Jumps'' Representation}
\label{sec:inference}

I develop a Gibbs sampling algorithm based on the Markov Process with
Failed Jumps representation, augmenting the data with the duration
variables $u$, the failed jump attempt count matrix, $Q$, as well
as additional auxiliary variables which we will define below.
In this representation the transition matrix is not modeled
directly, but is a function of the unscaled transition matrix $\pi$
and the similarity matrix $\phi$.  The full set of variables is
partitioned into three blocks: (1) $\{\gamma, \alpha, \beta, \pi\}$,
(2) $\{z, u, Q, \xi\}$, and (3) $\{\ell\}$, where $\xi$
is a placeholder for an additional set of auxiliary variables that will be introduced
below.  The variables in each block are sampled jointly 
conditioned on the other two blocks.  In some of the applications
described in later chapters, blocks (2) and (3) can be sampled
jointly, and if desired, the parameters of the $\phi$ function can be
given priors and sampled as a separate block.

Since we are representing the transition matrix of the Markov chain
explicitly, we approximate the stick-breaking process that produces
$\beta$ using a finite Dirichlet distribution with a number of 
components larger than we expect to need, forcing the remaining 
components to have zero weight.  

Let $J$ indicate the maximum number of states.  Then,
we approximate \eqref{eq:beta} with
\begin{equation}
  \label{eq:28}
  \beta \given \gamma \sim \mathrm{Dirichlet}(\gamma / J, \dots,
  \gamma / J)
\end{equation}
This distribution converges weakly to the Stick-Breaking Process as $J \to
\infty$ \cite{ishwaran2000markov}.  In practice, $J$ is large enough 
when the vast majority of the probability mass in $\beta$ is allocated 
to a strict subset of components, or when the latent state sequence 
$z$ never uses all $J$ available states, indicating that the data is 
well described by a number of states less than $J$.

\subsection{Sampling $\pi$, $\beta$, $\alpha$ and $\gamma$}
\label{sec:sampling-pi}

The joint conditional over $\gamma$, $\alpha$, $\beta$ and $\pi$
given $z$, $u$, $Q$, $\xi$ and $\theta$ will factor as
\begin{equation}
  \label{eq:46}
  p(\gamma, \alpha, \beta, \pi \given z, u, Q, \xi, \theta) = p(\gamma \given
  \xi) p(\alpha \given \xi) p(\beta \given \gamma, \xi) p(\pi
  \given \alpha, \beta, \theta, z)
\end{equation}
I will derive these four factors in reverse order.

\paragraph{Sampling $\pi$}

The entries in $\pi$ are conditionally independent given $\alpha$ and
$\beta$, so we have the prior
\begin{equation}
  \label{eq:47}
  p(\pi \given \beta, \alpha) = \prod_{j} \prod_{j'}
  \Gamma(\alpha\beta_{j'})^{-1}
  \pi_{jj'}^{\alpha\beta_{j'} - 1} \exp(-\pi_{jj'}),
\end{equation}
and the likelihood given augmented data $\{z, u, Q\}$ given by
\eqref{eq:joint-likelihood}.  Combining these, we have
\begin{align}
  \label{eq:61}
  p(\pi, z, u, Q \given \beta, \alpha, \theta) &=
  \prod_{j} u_j^{n_{j\cdot} + q_{j\cdot}
  - 1}\prod_{j'} 
  \Gamma(\alpha\beta_{j'})^{-1} \pi_{jj'}^{\alpha\beta_{j'} + n_{jj'}
    + q_{jj'} - 1} \\&\qquad \times e^{-(1 + u_j)
    \pi_{jj'}} \phi_{jj'}^{n_{jj'}} (1-\phi_{jj'})^{q_{jj'}} (q_{jj'}!)^{-1}
\end{align}
Conditioning on everything except $\pi$, we get
\begin{align}
  \label{eq:24}
  p(\pi \given Q, u, z, \beta, \alpha, \theta) &\propto \prod_j
  \prod_{j'} \pi_{jj'}^{\alpha\beta_{j'} + n_{jj'} + q_{jj'} - 1}
  \exp(-(1 + u_j)\pi_{jj'})
\end{align}
and thus we see that the $\pi_{jj'}$ are conditionally independent
given $u$, $z$ and $Q$, and distributed according to
\begin{align}
  \label{eq:25}
  \pi_{jj'} \given n_{jj'}, q_{jj'}, \beta_{j'}, \alpha \stackrel{ind}{\sim}
  \Gamm{\alpha\beta_{j'} + n_{jj'} + q_{jj'}}{1 + u_j}
\end{align}


\paragraph{Sampling $\beta$}
\label{sec:sampling-bbeta}

Consider the conditional distribution of $\beta$ having
integrated out $\pi$.  The prior density of $\beta$ from
\eqref{eq:28} is
\begin{equation}
  \label{eq:62}
  p(\beta \given \gamma) =
  \frac{\Gamma(\gamma)}{\Gamma(\frac{\gamma}{J})^J} \prod_{j}
  \beta_j^{\frac{\gamma}{J} - 1}
\end{equation}
After integrating out $\pi$ in \eqref{eq:61}, we have
\begin{align}
  p(z, u, Q \given \beta, \alpha, \gamma, \theta) &=
  \prod_{j=1}^J u_{j} ^{-1}
  \prod_{j'=1}^J u^{n_{jj'} + q_{jj'} - 1}(1 +
  u_j)^{-(\alpha\beta_{j'} + n_{jj'} + q_{jj'})}
  \\
  &\qquad \qquad \times \frac{\Gamma(\alpha\beta_{j'} + n_{jj'} +
    q_{jj'})}{\Gamma(\alpha\beta_{j'})} \phi_{jj'}^{n_{jj'}}(1-\phi_{jj'})^{q_{jj'}}
  (q_{jj'}!)^{-1} \\
  &= \prod_{j=1}^J \Gamma(n_{j\cdot})^{-1} u_j^{-1}(1+u_j)^{-\alpha}
  \left(\frac{u_j}{1+u_j}\right)^{n_{j\cdot} + q_{j\cdot}} \\ &\qquad
  \qquad \times \prod_{j' =
    1}^J \frac{\Gamma(\alpha\beta_{j'} + n_{jj'} +
    q_{jj'})}{\Gamma(\alpha\beta_{j'})} \phi_{jj'}^{n_{jj'}}(1-\phi_{jj'})^{q_{jj'}}
  (q_{jj'}!)^{-1}
\end{align}
 where we have used the fact that the $\beta_j$ sum to 1.  Therefore
\begin{align}
  p(\beta \given z, u, Q, \alpha, \gamma, \theta) &\propto \prod_{j=1}^J
  \beta_j^{\frac{\gamma}{J} - 1} \prod_{j'=1}^J \frac{\Gamma(\alpha\beta_{j'} +
    n_{jj'} + q_{jj'})}{\Gamma(\alpha\beta_{j'})}.
\end{align}

Following \citep{teh2006hierarchical}, we can write the ratios of Gamma functions
as polynomials in $\beta_j$, as
\begin{equation}
  \label{eq:31}
  p(\beta \given z, u, Q, \alpha, \gamma, \theta) \propto \prod_{j=1}^J
  \beta_j^{\frac{\gamma}{J} - 1} \prod_{j'=1}^{J} \sum_{m_{jj'} = 1}^{n_{jj'}}
  s(n_{jj'} + q_{jj'}, m_{jj'}) (\alpha \beta_{j'})^{m_{jj'}}
\end{equation}
where $s(m,n)$ is an unsigned Stirling number of the first kind, which
is used to represent the number of permutations of $n$ elements such that there are
$m$ distinct cycles.

This admits an augmented data representation, where we introduce a
random matrix $M = (m_{jj'})_{1 \leq j,j' \leq J}$, whose
entries are conditionally independent given $\beta$, $Q$ and $z$, with
\begin{equation}
  \label{eq:32}
  p(m_{jj'} = m \given \beta_{j'}, \alpha, n_{jj'}, q_{jj'}) =
  \frac{s(n_{jj'} + q_{jj'}, m) \alpha^{m}
    \beta_{j'}^{m}}{\sum_{m'=0}^{n_{jj'} + q_{jj'}} s(n_{jj'} +
  q_{jj'}, m') \alpha^{m'} \beta_{j'}^{m'}}
\end{equation}
for integer $m$ ranging between $0$ and $n_{jj'} + q_{jj'}$.  Note
that $s(n,0) = 0$ if $n > 0$, $s(0,0) = 1$, $s(0,m) = 0$ if $m >
0$, and we have the recurrence relation $s(n+1,m) = n s(n,m) + s(n,m-1)$, 
and so we could compute each of these coefficients explicitly;
however, it is typically simpler and more computationally efficient to sample from
this distribution than it is to enumerate its probabilities

For each $m_{jj'}$ we simply draw $n_{jj'}$ assignments of customers to tables 
according to the Chinese Restaurant Process and set $m_{jj'}$ to be
the number of distinct tables realized; that is, assign the first
customer to a table, setting $m_{jj'}$ to 1, and then, after $n$
customers are assigned, assign the $n+1$th customer to a new table
with probability $\alpha\beta_{j'} / (n + \alpha\beta_{j'})$, 
in which case we increment $m_{jj'}$, 
and to an existing table with probability $n / (n + \alpha)$, in which
case we do not increment $m_{jj'}$.  

% To see that this yields the
% distribution in \eqref{eq:32}, notice that, in order to end up with
% $m$ distinct tables, we need to draw the term with numerator
% $\alpha\beta_j'$ exactly $m$ times.  Irrespective of $m$, we have a
% product of terms of the form $n + \alpha\beta_{j'}$ for $n = 1, \dots,
% n_{jj'}$, and so the denominators can be absorbed into the constant of
% proportionality.  Then, all that remains is to sum the
% probabilities of different specific table assignments corresponding to
% the same value of $m$. This number corresponds exactly to $s(n_{jj'},m)$, which
% can be seen by inductively proving that the recurrence relation is the same.

% Suppose we have $n$ customers distributed among $m$ tables.  If $n
% > 0$ we must have $m$ at least 1, since the first customer will always
% start a new table, and so $s(n,0) = 0$.  If $n = 0$, then $m = 0$
% necessarily, so $s(0,0) = 1$ and $s(0,m) = 0$.
% Now, suppose that there are $s(n,m)$ ways to divide the first $n$ observations
% into $m$ tables.

Then, we have joint distribution
\begin{equation}
  \label{eq:33}
  p(\beta, M \given z, u, Q, \alpha, \gamma, \ell) \propto \prod_{j=1}^J
  \beta_j^{\frac{\gamma}{J} - 1} \prod_{j'=1}^{J} s(n_{jj'} + q_{jj'}, m_{jj'}) \alpha^{m_{jj'}} \beta_{j'}^{m_{jj'}}
\end{equation}
which yields \eqref{eq:31} when marginalized over $M$.  Again discarding
constants in $\beta$ and regrouping yields
\begin{equation}
  \label{eq:34}
  p(\beta \given M, z, u, \theta, \alpha, \gamma) \propto \prod_{{j'}=1}^J
  \beta_{j'}^{\frac{\gamma}{J} + m_{\cdot {j'}}- 1}
\end{equation}
which is Dirichlet:
\begin{equation}
  \label{eq:38}
  \beta \given M, \gamma \sim \mathrm{Dirichlet}(\frac{\gamma}{J} +
  m_{\cdot 1}, \dots, \frac{\gamma}{J} + m_{\cdot J})
\end{equation}

\paragraph{Sampling $\alpha$ and $\gamma$}
\label{sec:sampling-alpha}
Assume that $\alpha$ and $\gamma$ have Gamma priors, with
\begin{align}
  \label{eq:42}
  p(\alpha) &= \frac{b_{\alpha}^{a_{\alpha}}}{\Gamma(a_{\alpha})}
  \alpha^{a_{\alpha} - 1} \exp(-b_{\alpha}\alpha) \\
  p(\gamma) &= \frac{b_{\gamma}^{a_\gamma}}{\Gamma(a_{\gamma})}
  \gamma^{a_{\gamma - 1}} \exp(-b_{\gamma}\gamma)
\end{align}

Having integrated out $\pi$, we have
\begin{align}
  p(\beta, z, u, Q, M \given \alpha, \gamma, \theta) &=
  \frac{\Gamma(\gamma)}{\Gamma(\frac{\gamma}{J})^J} \alpha^{m_{\cdot\cdot}} \prod_{j=1}^J \beta_j^{\frac{\gamma}{J} +
    m_{\cdot j} - 1}\Gamma(n_{j\cdot})^{-1} u_j^{-1}(1+u_j)^{-\alpha}
  \left(\frac{u_j}{1+u_j}\right)^{n_{j\cdot} + q_{j\cdot}} \\ &\qquad
  \qquad \times \prod_{j' =
    1}^J s(n_{jj'} + q_{jj'}, m_{jj'}) \phi_{jj'}^{n_{jj'}}(1-\phi_{jj'})^{q_{jj'}}
  (q_{jj'}!)^{-1}
\end{align}
We can also integrate out $\beta$, to yield
\begin{align}
  p(z, u, Q, M \given \alpha, \gamma, \theta) &=
  \alpha^{m_{\cdot\cdot}} e^{-\sum_{j''} \log(1+u_{j''}) \alpha}
  \frac{\Gamma(\gamma)}{\Gamma(\gamma + m_{\cdot\cdot})} \\ &\qquad
  \qquad \times \prod_j
  \frac{\Gamma(\frac{\gamma}{J} + m_{\cdot
      j})}{\Gamma(\frac{\gamma}{J}) \Gamma(n_{j\cdot})} u_j^{-1}
  \left(\frac{u_j}{1+u_j}\right)^{n_{j\cdot} + q_{j\cdot}} \\ &\qquad
  \qquad \times \prod_{j' =
    1}^J s(n_{jj'} + q_{jj'}, m_{jj'}) \phi_{jj'}^{n_{jj'}}(1-\phi_{jj'})^{q_{jj'}}
  (q_{jj'}!)^{-1}
\end{align}
demonstrating that $\alpha$ and $\gamma$ are independent given $\theta$
and the augmented data, with
\begin{equation}
  \label{eq:43}
  p(\alpha \given z, u, Q, M, \theta) \propto
  \alpha^{a_{\alpha} + m_{\cdot\cdot}}\exp(-(b_\alpha + \sum_{j}\log(1+u_j))\alpha)
\end{equation}
and
\begin{align}
  \label{eq:8}
  p(\gamma \given z, u, Q, M, \theta) &\propto \gamma^{a_{\gamma - 1}}
  \exp(-b_{\gamma}\gamma) \frac{\Gamma(\gamma)\prod_{j=1}^J
    \Gamma(\frac{\gamma}{J} + m_{\cdot j})}{\Gamma(\frac{\gamma}{J})^J\Gamma(\gamma + m_{\cdot\cdot})}
\end{align}
So we see that
\begin{equation}
  \label{eq:44}
  \alpha \given z, u, Q, M, \theta \sim \Gamm{a_{\alpha}
    + m_{\cdot\cdot}}{b_\alpha + \sum_j\log(1+u_j)}
\end{equation}
To sample $\gamma$, we introduce a new set of auxiliary variables, $r = (r_1, \dots,
r_J)$ and $t$ with the following distributions:
\begin{align}
  \label{eq:9}
  p(r_{j'} = r \given m_{\cdot {j'}}, \gamma) &=
  \frac{\Gamma(\frac{\gamma}{J})}{\Gamma(\frac{\gamma}{J}
    + m_{\cdot {j'}})} s(m_{\cdot {j'}}, r)
    \left(\frac{\gamma}{J}\right)^r \qquad r  = 1, \dots, m_{\cdot j} \\
  p(t \given m_{\cdot\cdot} \gamma) &= \frac{\Gamma(\gamma +
    m_{\cdot\cdot})}{\Gamma(\gamma) \Gamma(m_{\cdot\cdot})} t^{\gamma
    - 1} (1-t)^{m_{\cdot\cdot} - 1} \qquad t \in (0,1)
\end{align}
so that
\begin{align}
  \label{eq:10}
  p(\gamma, r, t \given M) &\propto \gamma^{a_{\gamma - 1}}
  \exp(-b_{\gamma}\gamma) t^{\gamma - 1}(1-t)^{m_{\cdot\cdot}  - 1} \prod_{j'=1}^J s(m_{\cdot {j'}}, r_{j'})
  \left(\frac{\gamma}{J}\right)^{r_{j'}}
\end{align}
and
\begin{align}
  \label{eq:11}
  p(\gamma \given r, t) \propto \gamma^{a_\gamma +
    r_{\cdot} - 1} \exp(-(b_{\gamma} - \log(t)) \gamma),
\end{align}
which is to say
\begin{equation}
  \label{eq:18}
  \gamma \given r, t, z, u, Q, M, \theta \sim \Gamm{a_{\gamma} + r_{\cdot}}{b_{\gamma} - \log(t)}
\end{equation}

\subsubsection{Summary}

I have made the following additional assumptions about the generative
model in this section:
\begin{equation}
  \label{eq:100}
  \gamma \sim \Gamm{a_{\gamma}}{b_{\gamma}} \qquad \alpha \sim \Gamm{a_{\alpha}}{b_{\alpha}}
\end{equation}

The joint conditional over $\gamma$, $\alpha$, $\beta$ and $\pi$
given $z$, $u$, $Q$, $M$, $r$, $t$ and $\theta$ factors as
\begin{equation}
  \label{eq:46}
  p(\gamma, \alpha, \beta, \pi \given z, u, Q, r, t,
  \theta) = p(\gamma \given r, t) p(\alpha \given u, M) p(\beta
  \given \gamma, M) p(\pi \given \alpha, \beta, z, u, Q)
\end{equation}
where
\begin{align}
  \label{eq:64}
  \gamma \given r, t &\sim \Gamm{a_{\gamma} + r_{\cdot}}{b_{\gamma} -
    \log(t)} \\
  \alpha \given u, M &\sim \Gamm{a_{\alpha} +
    m_{\cdot\cdot}}{b_{\alpha} + \sum_j \log(1 + u_j)} \\
  \beta \given \gamma, M &\sim \mathrm{Dirichlet}(\frac{\gamma}{J} + m_{\cdot 1},
  \dots, \frac{\gamma}{J} + m_{\cdot J}) \\
  \pi_{jj'} \given \alpha, \beta_{j'}, z, u, Q
  &\stackrel{ind}{\sim} \Gamm{\alpha\beta_{j'} + n_{jj'} + q_{jj'}}{1 +
  u_j}
\end{align}


\subsection{Sampling $z$ and the auxiliary variables}
\label{sec:sampling-z_t}

The hidden state sequence, $z$, is sampled jointly with the auxiliary
variables, which consist of $u$, $M$, $Q$, $r$ and $t$.  The
joint conditional distribution of these variables is defined directly
by the generative model:
\begin{align}
  \label{eq:19}
  p(z, u, Q, M, r, t \given \pi, \beta, \alpha, \gamma,
  \ell) &= p(z \given \pi, \theta) p(u \given z, \pi, \ell) p(Q \given
  u, \pi, \ell) p(M \given
  z, Q, \alpha, \beta) \\
  &\qquad \times p(r \given
  \gamma, M) p(t \given \gamma, M)
\end{align}
Since we are representing the transition matrix explicitly, we can
sample the entire sequence $z$ at once with the forward-backward algorithm,
as in an ordinary HMM (or, if we are employing the HSMM variant
described in Sec. \ref{sec:an-hsmm-modification}, then we can use the
modified message passing scheme for HSMMs described by
\citet{johnson2013bayesian}).  
Having done this, we can sample $u$, $Q$, $M$,
$r$ and $t$ from their forward distributions.  To summarize,
we have
\begin{align}
  \label{eq:48}
  u_j \given z, \pi, \ell &\stackrel{ind}{\sim}
  \Gamm{n_{j\cdot}}{\sum_{j'} \pi_{jj'}\phi_{jj'}} \\
  q_{jj'} \given u_j, \pi_{jj'}, \phi_{jj'} &\stackrel{ind}{\sim}
  \Pois{u_j(1 - \phi_{jj'})\pi_{jj'}} \\
  m_{jj'} \given n_{jj'}, q_{jj'}, \beta_{j'}, \alpha &\stackrel{ind}{\sim}
  \frac{\Gamma(\alpha\beta_j)}{\Gamma(\alpha\beta_j + n_{jj'} +
    q_{jj'})}\sum_{m=1}^{n_{jj'} + q_{jj'}} s(n_{jj'} + q_{jj'}, m) \alpha^m \beta_{j'}^m \delta_{m}
  \\
  r_j \given m_{\cdot j}, \gamma &\stackrel{ind}{\sim}
  \frac{\Gamma(\frac{\gamma}{J})}{\Gamma(\frac{\gamma}{J} + m_{\cdot
      j})} \sum_{r=1}^{m_{j\cdot}} s(m_{\cdot j}, r)
  \left(\frac{\gamma}{J}\right)^r \delta_r \\
  t \given \gamma, M &\sim \Beta{\gamma}{m_{\cdot\cdot}}
\end{align}

\subsection{Sampling state and emission parameters}
\label{sec:sampling-eta}

The state parameters, $\theta$, influence the transition matrix,
$\pi$ and the auxiliary vector $q$ through the similarity matrix
matrix $\phi$, and also control the emission distributions.
We have likelihood factors
\begin{align}
  \label{eq:65}
  p(z, Q \given \theta) &\propto \prod_{j}\prod_{j'}
  \phi_{jj'}^{n_{jj'}}(1-\phi_{jj'})^{q_{jj'}} \\
  p(y \given z, \theta) &= \prod_{t=1}^T f(y_t; \theta_{z_t})
\end{align}
where, recall, $n_{jj'}$ counts the number of times that the state
sequence transitions from state $j$ to state $j'$, $\theta_j$ is the
part of $\ell_j$ that governs the emission distribution for state $j$,
and proportionality is with respect to variation in $\theta$.

The parameter space for the hidden states, 
the associated prior $H$ on $\theta$, and the similarity function
$\phi$, is application-specific, thus I now turn to three individual
applications with qualitatively different state spaces and emission distributions,
derive inference methods for each, and present experiments on
synthetic and real data.

\section{Use Cases}

In Chapter \ref{chapter:cocktail-party}, I describe an application of the
HaMMLeT model to a synthetic dataset designed to mimic a 
speaker diarization or blind source separation task. Here, each latent
state corresponds to a description of which of several sound sources
is active at a given time step, and where the observed data is a set
of signals picked up from several microphones distributed around the
room, each of which picks up all of the sound sources with varying
sensitivities, and the goal is to determine who is speaking when.  In
this application the latent states are represented as binary vectors,
similarity is based on Hamming distance between binary vectors, and
the emission model is a multivariate linear-Gaussian model.

In Chapter \ref{chapter:REDD} I describe a power disaggregation
application, in which the observation sequence consists the amount of power used
by a house at various times throughout a day, and the goal is to
separate the aggregated power signal into several signals
corresponding to a set of appliances (oven, air conditioning,
lighting, etc.).  Unlike the cocktail party setting, each latent
signal may have more than two levels, and so in addition to inferring
what appliances are on when, the model needs to discover how much
power each appliance uses in each of its latent states.  Here, latent
states are vectors where each entry is a categorical label, the number
of categories per channel is unknown, the similarity model is again
based on Hamming distance between binary vectors, and the emission 
model is linear-Gaussian.

Finally, in Chapter \ref{chapter:music} I describe an application to
unsupervised learning of musical structure.  The data is a sequence of
chords in a musical composition, represented as a single symbol, and
the goal is to infer a set of chord equivalence classes and a
transition model between equivalence classes.  Unlike in the previous
applications, similarity between states is modeled separately from the
emission distributions, so that $\ell_j = (\eta_j, \theta_j)$, where
$\phi$ is based on the $\eta_j$ part of the state parameters, and
$\theta_j$ is a disjoint set of parameters governing the emission distribution.

