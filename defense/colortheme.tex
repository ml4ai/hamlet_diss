% \usepackage{mathpazo}

% \mode<handout>{
% \pgfpagesuselayout{3 on 1 with notes}[letterpaper, border, shrink=0mm]
% }


\definecolor{paperbg}{HTML}{FDF6E3}
\definecolor{paperbgdark}{HTML}{EDE6CA}
\definecolor{solarred}{HTML}{DC322F}
\definecolor{solarlightorange}{HTML}{EBD6C3}
\definecolor{solarlightgreen}{HTML}{F3FFC8}
\definecolor{solarorange}{HTML}{C45B2A}
\definecolor{solaryellow}{HTML}{A57900}
\definecolor{solargreen}{HTML}{859900}
\definecolor{solarcyan}{HTML}{2AA198}
\definecolor{solarblue}{HTML}{268BD2}
\definecolor{solardarkblue}{HTML}{005BA2}
\definecolor{solarviolet}{HTML}{6C71C4}
\definecolor{solarmagenta}{HTML}{C05068}
\definecolor{solarbase2}{HTML}{EEE8D5}
\definecolor{solarbase1}{HTML}{93A1A1}
\definecolor{solarbase00}{HTML}{657B83}
\definecolor{solarbase01}{HTML}{586E75}
\definecolor{solarbase015}{HTML}{174652}
\definecolor{solarbase02}{HTML}{073642}
\definecolor{solarbase03}{HTML}{002B36}
\definecolor{functionblue}{HTML}{40A0B8}
\definecolor{solarlightblue}{HTML}{D9EEEF}

\usetheme{Singapore}
\usecolortheme[accent=cyan]{solarized}

\setbeamercolor{background canvas}{bg=white}
% \mode<handout>{
% \setbeamercolor{background canvas}{bg=white}
% }
\setbeamercolor{normal text}{fg=black}
% \usecolortheme[named=functionblue]{structure}
% \setbeamercolor{title in sidebar}{fg=solaryellow}
% \setbeamercolor{author in sidebar}{fg=solaryellow}
% \setbeamercolor{section in head/foot}{fg=paperbg}
% \setbeamercolor{section in sidebar}{fg=functionblue}
% \setbeamercolor{section in sidebar shaded}{fg=solarbase1}

% % \usecolortheme{orchid}
% \setbeamercolor{block body}{fg=solarbase01, bg=solarlightgreen}
% \setbeamercolor{block title}{fg=paperbg, bg=solargreen}
% \setbeamercolor{alertblock body}{fg=solarbase01, bg=solarlightorange}
% \setbeamercolor{alertblock title}{fg=paperbg, bg=solarorange}
% \setbeamertemplate{navigation symbols}{}

\AtBeginDocument{%
  % \setbeamercolor{frametitle}{fg=black}%
  \pgfdeclareverticalshading{beamer@headfade}{\paperwidth}
  {%
    color(0.00cm)=(solarbase2!50!paperbg);%
    color(0.00cm)=(solarbase2!50!paperbg)%
  }
  \setbeamercolor{section in head/foot}{fg=solaryellow}
} 

\newenvironment<>{attnblock}[1]{%
  \begin{actionenv}#2%  
      \def\insertblocktitle{#1}%
      \par%
      \mode<presentation>{%
        \setbeamercolor{block title}{fg=paperbg, bg=solarorange}
        \setbeamercolor{block body}{fg=solarbase01, bg=solarlightorange}
        \setbeamercolor{itemize item}{fg=solarorange}
        \setbeamercolor{itemize subitem}{fg=solarorange}
        }%
      \usebeamertemplate{block begin}}
{\par\usebeamertemplate{block end}\end{actionenv}}

\newenvironment<>{infoblock}[1]{%
  \begin{actionenv}#2%  
      \def\insertblocktitle{#1}%
      \par%
      \mode<presentation>{%
        \setbeamercolor{block title}{fg=paperbg, bg=solargreen}
        \setbeamercolor{block body}{fg=solarbase01, bg=solarlightgreen}
        \setbeamercolor{itemize item}{fg=solargreen}
        \setbeamercolor{itemize subitem}{fg=solargreen}
        \setbeamercolor{enumerate item}{fg=solargreen}
        }%
      \usebeamertemplate{block begin}}
{\par\usebeamertemplate{block end}\end{actionenv}}

\newenvironment<>{exblock}[1]{%
  \begin{actionenv}#2%  
      \def\insertblocktitle{#1}%
      \par%
      \mode<presentation>{%
        \setbeamercolor{block title}{fg=paperbg, bg=functionblue}
        \setbeamercolor{block body}{fg=solarbase01, bg=solarlightblue}
        \setbeamercolor{itemize item}{fg=functionblue}
        \setbeamercolor{itemize subitem}{fg=functionblue}
        }%
      \usebeamertemplate{block begin}}
{\par\usebeamertemplate{block end}\end{actionenv}}

% Double-line for start and end of epigraph.
\newcommand{\epiline}{\hrule \vskip -.2em \hrule}
% Massively humongous opening quotation mark.
\newcommand{\hugequote}{%
  \fontsize{42}{48}\selectfont \color{solarcyan} {\rm \textbf{``}}
  \vskip -.5em
}

% Beautify quotations.
\newcommand{\epigraph}[3]{%
  \bigskip
  \colorbox{solarbase2}{%
    \parbox{#1\textwidth}{%
    \epiline \vskip 1em {\hugequote} % \vskip -.5em
    % \parindent 2.2em
    {\rm #2}\begin{flushright}{\rm  #3}\end{flushright}
    \epiline
    }
  }
  \bigskip
}